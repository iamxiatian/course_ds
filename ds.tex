\documentclass[table, 13pt, slidestop,compress,mathserif]{beamer} 
% \documentclass[table, t, 13pt]{beamer}

% if use xelatex, use the following lines before lualatex
\usepackage{xeCJK,fontspec,xunicode,xltxtra, fancybox}
\setCJKmainfont{FZLanTingHei-L-GBK} %方正字体,也可以改成:微软雅黑
\setCJKfamilyfont{FZHeiR}{FZLanTingHeiS-R-GB}  
\setCJKsansfont{FZLanTingHeiS-EL-GB} 
\setCJKmonofont{Consolas}

% if use lualatex, use the following two lines.
%\usepackage{luatexja-fontspec}
%\setmainjfont{FZLanTingHei-L-GBK}

\setsansfont[Mapping=tex-text,LetterSpace=-1.25]{Ubuntu Light}
\setmonofont[Color=00663300]{Ubuntu Light}

%box tools
\usepackage{framed, xcolor} 
\usepackage{tabularx, colortbl, booktabs, multirow, makecell, longtable}
\usepackage{animate}

\usepackage{soul} %for strikeout

\usepackage{amsmath, amsfonts, amssymb} %amssymb for varnothing symbol
\usepackage{blkarray} % support complicated matrix
%\usepackage[thicklines]{cancel} %公式中通过斜线删除部分内容
\usepackage[b]{esvect} %vector

\usepackage{tcolorbox}
\tcbset{colback=white,colframe=orange!60,fonttitle=\bfseries,
coltitle=black}

\usepackage{caption, algorithm}
\usepackage[noend]{algpseudocode}

\usepackage{textpos}
\usepackage{adjustbox} %调整大小,例如缩放tikz绘图结果

\usepackage{pgfplots}
\usepackage{tikz, flowchart,times} % tikz绘图
\usepackage{smartdiagram}
\usetikzlibrary{decorations.pathreplacing}
\usetikzlibrary{decorations.markings}
\usetikzlibrary{calc, arrows, arrows.meta, shapes,snakes, shapes.geometric, positioning}
\usetikzlibrary{topaths}
\usetikzlibrary{mindmap, backgrounds}
\usetikzlibrary{shadings}
\usetikzlibrary{shadows}
\usetikzlibrary{graphs}
\usetikzlibrary{matrix}

\usepackage{tkz-graph}
\usepackage{forest}
\forestset{
  default preamble={
    for tree={circle,draw}
  },
  gappy tree/.style={
    for tree={
      circle,
      draw,
      s sep'+=10pt,
      fit=band,
    },
  },
  missed/.style={draw=none, no edge}
}

\usepackage{environ}
\NewEnviron{tikzthm}[1]{
\begin{tikzpicture}
\node [newthemsty] (box){\BODY};
\node[newthemstytitle, right=10pt] at (box.north west) {\textbf{#1}};
\end{tikzpicture}
}

% 新建幻灯片每一小节开始的样式环境
\NewEnviron{sectionbox}[1]{
  \begin{center}
    \tikzstyle{mybox} = [draw=blue, fill=green!20, very thick,
    rectangle, rounded corners, inner sep=10pt, inner ysep=20pt]
    \tikzstyle{fancytitle} =[fill=blue, text=white, ellipse]
    
    \vspace{1.0cm}
    \begin{tikzpicture}[transform shape, rotate=0, baseline=-3.5cm]
      \node [mybox] (box) {%
        \begin{minipage}[t!]{0.75\textwidth}
          \BODY
        \end{minipage}
      };
      \node[fancytitle] at (box.north) {#1};
    \end{tikzpicture}
  \end{center}
}%结束部分定义

\NewEnviron{outlinebox}[1]{
   \tikzstyle{mybox} = [draw=red, fill=blue!5, very thick,
  rectangle, rounded corners, inner sep=10pt, inner ysep=20pt]
  \tikzstyle{fancytitle} =[fill=red, text=white]
  \begin{center}
    \begin{tikzpicture}
      \node [mybox] (box){%
        \begin{minipage}{0.80\textwidth}
          \BODY
        \end{minipage}
      };
      \node[fancytitle, right=10pt] at (box.north west) { #1 };
    \end{tikzpicture}
  \end{center}
}

\usepackage{tcolorbox}
\tcbuselibrary{skins}

\NewEnviron{infobox}[1]{
  \begin{center}
    \begin{tcolorbox}[width=0.9\textwidth, title={#1},enhanced, colframe=red,colback=white,
      arc=1mm,colbacktitle=red!10,
      fonttitle=\bfseries,coltitle=black,
      attach boxed title to top left=
      {xshift=3.2mm,yshift=-0.50mm},
      boxed title style={skin=enhancedfirst jigsaw,
        size=small,arc=1mm,bottom=-1mm,
        interior style={fill=none,
          top color=red!30!white,
          bottom color=red!20!white}}]
      \BODY
    \end{tcolorbox}
  \end{center}
}


\usepackage{minted} %compile: lualatex/xelatex -shell-escape spark.tex
\setminted{encoding=utf-8} %注意字体,如果设置了CJKmonofont,会出现乱码
\usemintedstyle{tango}

%\usepackage[outputdir={tempdot/}]{dot2texi}

\usetheme{Madrid} 
\usecolortheme{crane} 

\setbeamertemplate{items}[ball] 
\setbeamertemplate{blocks}[rounded][shadow=true] 

\usefonttheme[onlymath]{serif}

%\definecolor{red(ncs)}{rgb}{0.77, 0.01, 0.2}
%\definecolor{champagne}{rgb}{0.97, 0.91, 0.81}
\definecolor{coolblack}{rgb}{0.0, 0.18, 0.39}
\definecolor{vanilla}{rgb}{0.95, 0.9, 0.67}

\usepackage[ampersand]{easylist}
\newcommand\easyitem{\ListProperties(Hide=100, Hang=true, Progressive=3ex,
  Style*=\color{orange}$\bullet$ ,
  Style2*=\color{orange}$\ast$ ,
  Style3*=\color{orange}$\circ$ ,
  Style4*=\tiny$\blacksquare$, Space=-.5em, Space*=-.5em)}

\setbeamertemplate{itemize item}{\color{orange}$\bullet$}
\setbeamertemplate{itemize subitem}{\tiny\raise1.5pt\hbox{\donotcoloroutermaths$\blacktriangleright$}}
\setbeamertemplate{itemize subsubitem}{\tiny\raise1.5pt\hbox{\donotcoloroutermaths$\blacktriangleright$}}
\setbeamertemplate{enumerate item}{\insertenumlabel.}
\setbeamertemplate{enumerate subitem}{\insertenumlabel.\insertsubenumlabel}
\setbeamertemplate{enumerate subsubitem}{\insertenumlabel.\insertsubenumlabel.\insertsubsubenumlabel}
\setbeamertemplate{enumerate mini template}{\insertenumlabel}

\parskip=3mm
\parindent=15pt
\linespread{1.2}

%自定义的一些命令,方便使用
\newcommand*\circled[2][black]{\tikz[baseline=(char.base)]{
  \node[shape=circle,draw=#1,inner sep=1.5pt] (char) {#2};}}

\newcommand{\cjkbold}{\color[rgb]{0.29, 0.0, 0.51} \CJKfamily{FZHei}}  %http://latexcolor.com/

\newcommand{\cjkem}{\CJKfamily{FZHeiR}} 
\renewcommand{\em}[1]{\color{red} #1}

%\XeTeXlinebreaklocale "zh"  
%\XeTeXlinebreakskip = 0pt plus 1pt 

\hypersetup{
  pdftitle={Data Structure},
  pdfsubject={Data Structure},
  pdfkeywords={Data Structure},
  pdfproducer={LaTeX},
  pdfcreator={XeLaTeX}
}

\setbeamercolor{title}{fg=coolblack, bg=orange!30}
\setbeamercolor{frametitle}{fg=coolblack, bg=vanilla!0}

\setbeamercolor{palette primary}{fg=black, bg=gray!15!white}
\setbeamercolor{palette secondary}{fg=black, bg=gray!10!white}
\setbeamercolor{palette tertiary}{fg=black, bg=gray!15!white}

\addtobeamertemplate{frametitle}{}{%
\begin{textblock*}{1.0\paperwidth}(-.001\textwidth,0cm)
%\tikz{\draw[orange!70!yellow, line width=1.2] (-1cm,0cm) -- (0.5\textwidth,0cm);\draw[orange!70!yellow,yshift=-0.5] (0.5\textwidth,0cm) -- (0.7\textwidth,0cm);}
\end{textblock*}
\begin{textblock*}{100mm}(.85\textwidth,-1cm)
\includegraphics[height=1.2cm,width=1.2cm]{ruc_logo.png}
\end{textblock*}
}

%gets rid of bottom navigation bars
\setbeamertemplate{footline}[page number]{}

%gets rid of navigation symbols
\setbeamertemplate{navigation symbols}{}

\begin{document}

% \logo{\includegraphics[width=1.0cm,height=1.0cm]{figure/ruc.jpg}}
\title{数据结构 \\ Data Structure}
\author{Xia Tian \\ Email: xiat(at)ruc.edu.cn }
\institute{Renmin University of China }
\date{
  % \today{}
}
% \date[\initclock\tdtime]{\today}  
\frame{\titlepage}

\section{Tree}


\begin{frame}[fragile]{树和二叉树}
  树型结构是结点之间有分支,并且具有层次关系的结构,类似于自然界中的树。树有很多应
  用,比如Unix等操作系统中的目录结构。
\end{frame}

\begin{frame}[fragile]
  \frametitle{例子}
\begin{forest}
 [CEO, for tree={rectangle, minimum width=2cm}, fill=red!10
    [CFO [财务人员] ]
    [CTO [工程师] ]
    [CMO [销售人员] ]
 ]
 \node at (current bounding box.south)
 [below=1ex,draw,cloud,aspect=6,cloud puffs=30]
 {\emph{Simple Company Hierarchy}};
\end{forest}
\end{frame}

\begin{frame}[fragile, plain]
  \scalebox{0.7}{
    \begin{forest}
      [学院, for tree={draw=none, rectangle, minimum width=1cm}, fill=red!10, circle
       [社会学部, grow=west [信息资源管理学院, fill=red!10] [新闻学院] [农业与农村发展学院] [社会与人口学院]
       [公共管理学院] [教育学院]]
       [$\cdots$]
       [ 人文学部,grow=east [哲学院] [文学院] [历史学院] [国学院] [艺术学院] [外国语学院] [清史研究所]]
       ]
       \node at (current bounding box.south)
       [below=1ex,draw,cloud,aspect=6,cloud puffs=30]
       {\emph{人民大学学院设置}};
    \end{forest}
  }
\end{frame}

\begin{frame}[fragile]{内容}
  \begin{easylist} \easyitem
    & 树的基本术语
    & 二叉树
    & 遍历二叉树与线索二叉树
    & 树和森林
    & 哈夫曼树
  \end{easylist}
\end{frame}

\subsection{基本术语}

\begin{frame}[fragile]
  \frametitle{树(TREE)}树(Tree)是$n(n \geq 0)$个结点的有限集$T$。 $T$为空时称为空
  树。当$n>0$时,树有且仅有一个特定的称为根(Root)的结点,其余结点可分为$m(m \geq
  0)$个互不相交的子集$T_1, T_2, \cdots, T_m$,其中每个子集又是一棵树,称为子
  树(Subtree)。
  \begin{enumerate}
  \item 各子树是互不相交的集合。
  \item 除根结点,其它结点有唯一前驱。
  \item   一个结点可以有零个或多个后继。
  \end{enumerate}

  \begin{forest}
    [R, for tree={color=white,fill=black}, fill=red!85
    [A [C] [D] [E]]
    [B [F]]
    ]
  \end{forest}
\end{frame}

\begin{frame}[fragile]
  \frametitle{判断哪些是树结构}
  \includegraphics[width=0.3\textwidth]{dot/tree-judge1.pdf} ~~~~~
  \pause
  \includegraphics[width=0.4\textwidth]{dot/tree-judge2.pdf}
\end{frame}

\begin{frame}[fragile]
  \frametitle{判断哪些是树结构}
  \includegraphics[width=0.35\textwidth]{dot/tree-judge3.pdf} ~~~~~
  \pause
  \includegraphics[width=0.4\textwidth]{dot/tree-judge4.pdf}
\end{frame}

\begin{frame}[fragile]
  \frametitle{树的表示形式}
  \includegraphics[width=0.36\textwidth]{dot/tree-represent1.pdf}  \pause
  \scalebox{0.75}{
    \begin{tikzpicture}[b/.style={fill=black!50},n/.style={minimum width=1cm}]
      \draw node[n] (a) {A} node[b, right=0 of a, minimum width=5cm, fill=red!50]{};	

      \draw node[minimum width=0.5cm, below=0.1 of a](bh){} 
      node[n, right=0 of bh] (b) {B} node[b,fill=blue!50, minimum width=4.3cm,right=0 of b]{};	

      \draw node[minimum width=2cm, below=0.2 of bh](dh){} 
      node[n, right=0 of dh] (d) {D} node[b, minimum width=3.5cm,right=0 of d]{};	

      \draw node[minimum width=3.5cm, below=0.2 of dh](ih){} 
      node[n, right=0 of ih] (i) {I} node[b, fill=green!50, minimum width=2.8cm,right=0 of i]{};	

      \draw node[minimum width=3.5cm, below=0.2 of ih](jh){} 
      node[n, right=0 of jh] (j) {J} node[b, fill=green!50, minimum width=2.8cm,right=0 of j]{};	

      \draw node[minimum width=2cm, below=1.2 of dh](eh){} 
      node[n, right=0 of eh] (e) {E} node[b, minimum width=3.5cm,right=0 of e]{};	

      \draw node[minimum width=2cm, below=1.8 of dh](fh){} 
      node[n, right=0 of fh] (f) {F} node[b, minimum width=3.5cm,right=0 of f]{};		


      \draw node[minimum width=0.5cm, below=3.2 of a](ch){} 
      node[n, right=0 of ch] (c) {C} node[b, fill=blue!50, minimum width=4.3cm,right=0 of c]{};	

      \draw node[minimum width=2cm, below=2.8 of dh](gh){} 
      node[n, right=0 of gh] (g) {G} node[b, minimum width=3.5cm,right=0 of g]{};	
      \draw node[minimum width=2cm, below=3.2 of dh](hh){} 
      node[n, right=0 of hh] (h) {H} node[b, minimum width=3.5cm,right=0 of h]{};	

      \draw node[below=0.1 of h] {凹入表表示法};
    \end{tikzpicture}
      } \pause
  \begin{columns}[t]
    \begin{column}{0.4\textwidth}
      \centering
      \vspace{0pt}
      (A(B(D(I,J),E, F),C(G,H)))
      
      广义表表示 \\
      
      \pause
    \end{column}
    \begin{column}{0.5\textwidth}
\scalebox{0.65}{
    \begin{tikzpicture}[n/.style={ellipse,draw}]
      \draw node[n, minimum width=7.8cm, minimum height=3.5cm, fill=red!5]{}
      node[n, minimum width=4.5cm, minimum height=2.5cm, xshift=-1.2cm, fill=blue!5]{}
      node[n, minimum width=2.5cm, minimum height=1.8cm, xshift=2.5cm, fill=blue!5]{}
      node[n, minimum width=2cm, minimum height=1.8cm, xshift=-2cm, fill=yellow!5]{}
      node[n, circle,xshift=-2.5cm, fill=green!5]{I}
      node[n, circle,xshift=-1.7cm, fill=green!5]{J}
      node[n, circle,xshift=-0.5cm, fill=yellow!5]{E}
      node[n, circle,xshift=0.4cm, fill=yellow!5]{F}
      node[n, circle,xshift=2cm, fill=yellow!5]{G}
      node[n, circle,xshift=3cm, fill=yellow!5]{H}
      node[yshift=1.35cm]{A}
      node[xshift=-0.8cm,yshift=0.8cm]{B}
      node[xshift=2.5cm, yshift=0.6cm]{C}
      node[xshift=-2cm,yshift=0.6cm]{D};
      \draw node[yshift=-2.2cm] {嵌套集合表示};
    \end{tikzpicture}
  }
  
    \end{column}
  \end{columns}
\end{frame}

\begin{frame}[fragile]{}
\begin{easylist} \easyitem

\end{easylist}
\end{frame}



%\section{Graph}

\begin{frame}[fragile]{Graph}
  \begin{adjustbox}{max totalsize={.9\textwidth}{.7\textheight},center}
    \tikzstyle{every node}=[circle, draw, fill=black!50,
    inner sep=0pt, minimum width=4pt]
    % Tutte's 8-cage
    \begin{tikzpicture}[thick,scale=0.8]
      \draw \foreach \x in {0,36,...,324}
      {
        (\x:2) node {}  -- (\x+108:2)
        (\x-10:3) node {} -- (\x+5:4)
        (\x-10:3) -- (\x+36:2)
        (\x-10:3) --(\x+170:3)
        (\x+5:4) node {} -- (\x+41:4)
      };
    \end{tikzpicture}\quad

    % The largest 3-regular graph of diameter 3
    \begin{tikzpicture}[thick,scale=0.8]%
      \draw \foreach \x in {18,90,...,306} {
        (\x:4) node{} -- (\x+72:4)
        (\x:4) -- (\x:3) node{}
        (\x:3) -- (\x+15:2) node{}
        (\x:3) -- (\x-15:2) node{}
        (\x+15:2) -- (\x+144-15:2)
        (\x-15:2) -- (\x+144+15:2)
      };
    \end{tikzpicture}
  \end{adjustbox}
\end{frame}

\begin{frame}[fragile]{Content}
  \begin{easylist} \easyitem
    & 图的定义
    & 图的存储表示
    & 图的遍历
    & 图的连通性
  \end{easylist}
\end{frame}

\begin{frame}[fragile]
  \frametitle{图(Graph)}
  \begin{itemize}
  \item 图$G=(V, E)$, $V$是顶点(Vertex)集合,$E$是边/弧(Edge/Arc)的集合.
  \item 顶点的度、出度和入度
  \end{itemize}

  \begin{columns}[T]
    \column{0.5\textwidth}
    有向图:
    
    \begin{tikzpicture}[scale=1.0]
      \GraphInit[vstyle=Art]
      \Vertex{A}
      \Vertex[x=4,y=0]{B}
      \Vertex[x=0,y=2]{C}
      \Vertex[x=4,y=2]{D}
      \Edge[style={-Latex}](A)(D)

      \Edges[style={-Latex}](A,B,C, D)
      \Edges[style={-Latex}](A, C)
    \end{tikzpicture}
    
    \column{0.5\textwidth}
    无向图:
    
    \begin{tikzpicture}[scale=1.0]
      \GraphInit[vstyle=Art]
      \Vertex{A}
      \Vertex[x=4,y=0]{B}
      \Vertex[x=0,y=2]{C}
      \Vertex[x=4,y=2]{D}
      \Edge[style={}](A)(D)

      \Edges[style={}](A,B,C, D)
      \Edges[style={}](A, C)
    \end{tikzpicture}
  \end{columns}
\end{frame}

\begin{frame}[fragile]
  \frametitle{图的相关概念}
  \includegraphics[width=0.9\textwidth]{figs/graph_concept.png}
\end{frame}

\begin{frame}[plain]
~  
\end{frame}

\begin{frame}[fragile]
  \frametitle{图的存储}
  如何表达下图的信息?
  \begin{columns}
    \column{0.4\textwidth}
    有向图:
    
    \begin{tikzpicture}[scale=1.3]
      \GraphInit[vstyle=Normal]
      \begin{scope}[rotate=180]
        \Vertices{circle}{$v_1$, $v_2$, $v_3$, $v_4$}
      \end{scope}
      \Edges[style={-Latex}]($v_1$,  $v_3$, $v_4$, $v_1$, $v_2$)
    \end{tikzpicture}
    
    \column{0.4\textwidth}
    无向图:
    
    \begin{tikzpicture}[scale=1.5]
      \GraphInit[vstyle=Normal]
      \begin{scope}[rotate=180]
        \Vertices{circle}{$v_1$, $v_2$, $v_3$, $v_4$}
      \end{scope}
      \Vertex{$v_5$}
      \Edges($v_1$, $v_2$, $v_3$, $v_4$, $v_1$)
      \Edges($v_2$, $v_5$, $v_3$)
    \end{tikzpicture}
  \end{columns}
  
  \pause
  \begin{itemize}
  \item 可用邻接矩阵表达顶点及其关系。
  \end{itemize}
\end{frame}


\begin{frame}[fragile]
  \frametitle{图的存储}
  \small
  \begin{columns}[T]
    \column{0.4\textwidth}
    \begin{tikzpicture}[scale=1]
      \GraphInit[vstyle=Normal]
      \begin{scope}[rotate=180]
        \Vertices{circle}{$v_1$, $v_2$, $v_3$, $v_4$}
      \end{scope}
      \Edges[style={-Latex}]($v_1$,  $v_3$, $v_4$, $v_1$, $v_2$)
    \end{tikzpicture}
    \[
      \begin{blockarray}{ccccc}
         & v_1 & v_2 & v_3 & v_4 \\
        \begin{block}{c (c c c c)}
          v_1 & 0 & 1 & 1 & 0 \\
          v_2 & 0 & 0 & 0 & 0 \\
          v_3 & 0 & 0 & 0 & 1 \\
          v_4 & 1 & 0 & 0 & 0 \\
        \end{block}
      \end{blockarray}
    \]

    \column{0.4\textwidth}
    \begin{tikzpicture}[scale=1.0]
      \GraphInit[vstyle=Normal]
      \begin{scope}[rotate=180]
        \Vertices{circle}{$v_1$, $v_2$, $v_3$, $v_4$}
      \end{scope}
      \Vertex{$v_5$}
      \Edges($v_1$, $v_2$, $v_3$, $v_4$, $v_1$)
      \Edges($v_2$, $v_5$, $v_3$)
    \end{tikzpicture}

    \[
      \begin{blockarray}{cccccc}
         & v_1 & v_2 & v_3 & v_4 & v_5 \\
        \begin{block}{c (c c c c c)}
          v_1 & 0 & 1 & 0 & 1 & 0 \\
          v_2 & 1 & 0 & 1 & 0 & 1 \\
          v_3 & 0 & 1 & 0 & 1 & 1 \\
          v_4 & 1 & 0 & 1 & 0 & 0 \\
          v_5 & 0 & 1 & 1 & 0 & 0 \\
        \end{block}
      \end{blockarray}
    \]    
  \end{columns}
  
  \begin{itemize}
  \item 根据邻接矩阵,如何判断各顶点的度?
  \end{itemize}
\end{frame}

\begin{frame}[fragile]
  \frametitle{有向图的连续存储方式:邻接矩阵}
  \begin{columns}[T]
    \column{0.4\textwidth}
    \begin{tikzpicture}[scale=1.4]
      \GraphInit[vstyle=Normal]
      \begin{scope}[rotate=180]
        \Vertices{circle}{$v_1$, $v_2$, $v_3$, $v_4$}
      \end{scope}
      \Edges[style={-Latex}]($v_1$,  $v_3$, $v_4$, $v_1$, $v_2$)
    \end{tikzpicture}
    \[
      \begin{blockarray}{ccccc}
        & v_1 & v_2 & v_3 & v_4 \\
        \begin{block}{c (c c c c)}
          v_1 & 0 & 1 & 1 & 0 \\
          v_2 & 0 & 0 & 0 & 0 \\
          v_3 & 0 & 0 & 0 & 1 \\
          v_4 & 1 & 0 & 0 & 0 \\
        \end{block}
      \end{blockarray}
    \]
    
    \column{0.6\textwidth}
    \begin{itemize}
    \item 建立二维数组$A[n][n]$, $n=|V|$
    \item 另需存放$n$个顶点信息
    \end{itemize}
  \end{columns}
\end{frame}

\begin{frame}[fragile]
  \frametitle{网的邻接矩阵}
  \begin{columns}[T]
    \column{0.4\textwidth}
    \begin{tikzpicture}[scale=1.8]
      \GraphInit[vstyle=Normal]
      \begin{scope}[rotate=180]
        \Vertices{circle}{v1, v2, v3, v4}
      \end{scope}
      \Edge[label = 8, style={-Latex}](v1)(v2)
      \Edge[label = 3, style={-Latex}](v1)(v3)
      \Edge[label = 5, style={-Latex}](v4)(v1)
      \Edge[label = 1, style={-Latex}](v3)(v4)
    \end{tikzpicture}
    \[
      \begin{blockarray}{ccccc}
        & v_1 & v_2 & v_3 & v_4 \\
        \begin{block}{c (c c c c)}
          v_1 & \infty & 8 & 3 & \infty \\
          v_2 & \infty & \infty & \infty & \infty \\
          v_3 & \infty & \infty & \infty & 1 \\
          v_4 & 5 & \infty & \infty & \infty \\
        \end{block}
      \end{blockarray}
    \]
    
    \column{0.6\textwidth}
    \begin{itemize}
    \item 有些图的边带有权重(常用来表示成本、距离、时间等), 这样的图称为:{\color{red} 网}。
    \item 网的邻接矩阵表达权重,没有边的顶点之间的权重默认为$\infty$
    \item 邻接矩阵表示方法非常直观、简单,但是会有什么问题? \pause
    \item 现实中的图经常对应稀疏矩阵,在这样情形下会有很大空间浪费.
    \end{itemize}
  \end{columns}  
\end{frame}

\begin{frame}[fragile]
  \frametitle{邻接表 (Adjacency List) -- 无向图}
  \begin{columns}[T]
    \column{0.35\textwidth}
    \begin{tikzpicture}[scale=1.3]
      \GraphInit[vstyle=Normal]
      \begin{scope}[rotate=90]
        \Vertices{circle}{v1, v2, v3, v4}
      \end{scope}
      \Edges(v3, v1, v2, v4)
    \end{tikzpicture}
    
    \column{0.6\textwidth}
    \scalebox{0.8}{
      \begin{tikzpicture}[ n/.style={minimum height=0.8cm, minimum width=1cm},
        n2/.style={minimum height=0.6cm, minimum width=1cm, fill=red!5},
        e/.style={->, very thick}]
        \draw node[n] (n0) {索引} node[n,right=0 of n0](label) {头节点};

        \foreach \x  [evaluate = \x as \xp using int(\x-1)] in {1, ..., 4}
        \draw node[n, below=0 of n\xp](n\x) {$\xp$};

        \foreach \x/\y/\z in {1/A/,2/B/,3/C/,4/D/}
        \draw node[n, right=0 of n\x, draw, fill=yellow!10](\y) {$v_\x$} node[n, right=0 of \y, draw] (P\y) {\z};


        \draw node[n2, draw, right=of PA] (c11) {2} node[n2, draw, right=0 of c11] (c12) {};
        \draw node[n2, draw, right=of c12] (c21) {1} node[n2, draw, right=0 of c21] (c22) {$\wedge$};
        \draw[e] (PA.center) -- (c11);
        \draw[e] (c12.center)--(c21);				

        \draw node[n2, draw, right=of PB] (c11) {3} node[n2, draw, right=0 of c11] (c12) {};
        \draw node[n2, draw, right=of c12] (c21) {0} node[n2, draw, right=0 of c21] (c22) {$\wedge$};
        \draw[e] (PB.center) -- (c11) (c12.center)--(c21);				

        \draw node[n2, draw, right=of PC] (c11) {0} node[n2, draw, right=0 of c11] (c12) {$\wedge$};
        \draw[e] (PC.center) -- (c11);

        \draw node[n2, draw, right=of PD] (c11) {1} node[n2, draw, right=0 of c11] (c12) {$\wedge$};
        \draw[e] (PD.center) -- (c11);
      \end{tikzpicture}
    }
  \end{columns}

  \begin{itemize}
  \item 无向图的邻接表:同一个顶点发出的边链接在同一个边链表中,便于确定顶点的度
  \item 需要$n$个头结点, $2e$个表结点
  \end{itemize}
\end{frame}

\begin{frame}[fragile]
  \frametitle{邻接表--有向图}
  \begin{columns}[T]
    \column{0.4\textwidth}
    \begin{tikzpicture}[scale=1.5]
      \GraphInit[vstyle=Normal]
      \begin{scope}[rotate=135]
        \Vertices{circle}{A, B, C, E}
      \end{scope}
      \Vertex{D}
      \Edges[style={-Latex}](A, B, C, D, E, A)
      \Edge[style={-Latex}](A)(D)	
    \end{tikzpicture}
    
    邻接表,便于确定节点出度

    \scalebox{0.7}{
      \begin{tikzpicture}[ n/.style={minimum height=0.8cm, minimum width=1cm},
        n2/.style={minimum height=0.6cm, minimum width=1cm, fill=red!5},
        e/.style={->, very thick}]
        \draw node[n] (n0) {索引} node[n,right=0 of n0](label) {头节点};

        \foreach \x  [evaluate = \x as \xp using int(\x-1)] in {1, ..., 5}
        \draw node[n, below=0 of n\xp](n\x) {$\xp$};

        \foreach \x/\y/\z in {1/A/,2/B/,3/C/,4/D/, 5/E/}
        \draw node[n, right=0 of n\x, draw, fill=yellow!10](\y) {$\y$} node[n, right=0 of \y, draw] (P\y) {\z};


        \draw node[n2, draw, right=of PA] (c11) {3} node[n2, draw, right=0 of c11] (c12) {};
        \draw node[n2, draw, right=of c12] (c21) {1} node[n2, draw, right=0 of c21] (c22) {$\wedge$};
        \draw[e] (PA.center) -- (c11);
        \draw[e] (c12.center)--(c21);				

        \draw node[n2, draw, right=of PB] (c11) {2} node[n2, draw, right=0 of c11] (c12) {$\wedge$};
        \draw[e] (PB.center) -- (c11);

        \draw node[n2, draw, right=of PC] (c11) {3} node[n2, draw, right=0 of c11] (c12) {$\wedge$};
        \draw[e] (PC.center) -- (c11);

        \draw node[n2, draw, right=of PD] (c11) {4} node[n2, draw, right=0 of c11] (c12) {$\wedge$};
        \draw[e] (PD.center) -- (c11);

        \draw node[n2, draw, right=of PE] (c11) {0} node[n2, draw, right=0 of c11] (c12) {$\wedge$};
        \draw[e] (PE.center) -- (c11);
      \end{tikzpicture}
    }
    \pause
    
    \column{0.6\textwidth}
    逆邻接表,便于确定节点入度
    
    \scalebox{0.7}{
      \begin{tikzpicture}[ n/.style={minimum height=0.8cm, minimum width=1cm},
        n2/.style={minimum height=0.6cm, minimum width=1cm, fill=red!5},
        e/.style={->, very thick}]
        \draw node[n] (n0) {索引} node[n,right=0 of n0](label) {头节点};

        \foreach \x  [evaluate = \x as \xp using int(\x-1)] in {1, ..., 5}
        \draw node[n, below=0 of n\xp](n\x) {$\xp$};

        \foreach \x/\y/\z in {1/A/,2/B/,3/C/,4/D/, 5/E/}
        \draw node[n, right=0 of n\x, draw, fill=yellow!10](\y) {$\y$} node[n, right=0 of \y, draw] (P\y) {\z};

        \draw node[n2, draw, right=of PA] (c11) {4} node[n2, draw, right=0 of c11] (c12) {$\wedge$};
        \draw[e] (PA.center) -- (c11);

        \draw node[n2, draw, right=of PB] (c11) {0} node[n2, draw, right=0 of c11] (c12) {$\wedge$};
        \draw[e] (PB.center) -- (c11);

        \draw node[n2, draw, right=of PC] (c11) {1} node[n2, draw, right=0 of c11] (c12) {$\wedge$};
        \draw[e] (PC.center) -- (c11);

        \draw node[n2, draw, right=of PD] (c11) {2} node[n2, draw, right=0 of c11] (c12) {};
        \draw node[n2, draw, right=of c12] (c21) {0} node[n2, draw, right=0 of c21] (c22) {$\wedge$};
        \draw[e] (PD.center) -- (c11);
        \draw[e] (c12.center)--(c21);		

        \draw node[n2, draw, right=of PE] (c11) {3} node[n2, draw, right=0 of c11] (c12) {$\wedge$};
        \draw[e] (PE.center) -- (c11);
      \end{tikzpicture}
    }    
  \end{columns}
\end{frame}

\begin{frame}[fragile]
  \frametitle{邻接表--权重处理}
  \scalebox{0.7}{
    \begin{tikzpicture}[scale=2]
      \GraphInit[vstyle=Normal]
      \begin{scope}[rotate=225]
        \Vertices{circle}{A, B, D, C}
      \end{scope}

      \Edge[style={-Latex}, label=1](A)(B)
      \Edge[style={-Latex, pos=0.2}, label=4](A)(D)

      \Edge[style={-Latex, pos=0.2, bend right}, label=9](B)(C)
      \Edge[style={-Latex}, label=2](B)(D)

      \Edge[style={-Latex}, label=3](C)(A)
      \Edge[style={-Latex, pos=0.2, bend right=10}, label=5](C)(B)
      \Edge[style={-Latex}, label=8](C)(D)

      \Edge[style={-Latex, bend right=40}, label=6](D)(C)
    \end{tikzpicture}
  }

  \scalebox{0.7}{
    \begin{tikzpicture}[ n/.style={minimum height=0.8cm, minimum width=1cm},
      n2/.style={minimum height=0.6cm, minimum width=1cm, fill=red!5},
      e/.style={->, very thick}]
      \draw node[n] (n0) {索引} node[n,right=0 of n0](label) {头节点};

      \foreach \x  [evaluate = \x as \xp using int(\x-1)] in {1, ..., 4}
      \draw node[n, below=0 of n\xp](n\x) {$\xp$};

      \foreach \x/\y/\z in {1/A/,2/B/,3/C/,4/D/}
      \draw node[n, right=0 of n\x, draw, fill=yellow!10](\y) {$\y$} node[n, right=0 of \y, draw] (P\y) {\z};

      \draw node[n2, draw, right=of PA] (c11) {1} node[n2, draw, right=0 of c11,fill=blue!10] (c12) {1} node[n2, draw, right=0 of c12] (c13) {};
      \draw node[n2, draw, right=of c13] (c21) {3}  node[n2, draw, right=0 of c21,fill=blue!10] (c22) {4} node[n2, draw, right=0 of c22] (c23) {$\wedge$};
      \draw[e] (PA.center) -- (c11);
      \draw[e] (c13.center) -- (c21);

      \draw node[above=of c11](tip1) {边的终点} node[right=of tip1](tip2){权重};
      \path[draw, ->] (tip1) edge (c11) (tip2) edge[bend right] (c12);
      
      \draw node[n2, draw, right=of PB] (c11) {3} node[n2, draw, right=0 of c11,fill=blue!10] (c12) {2} node[n2, draw, right=0 of c12] (c13) {};
      \draw node[n2, draw, right=of c13] (c21) {2}  node[n2, draw, right=0 of c21,fill=blue!10] (c22) {9} node[n2, draw, right=0 of c22] (c23) {$\wedge$};
      \draw[e] (PB.center) -- (c11);
      \draw[e] (c13.center) -- (c21);

      \draw node[n2, draw, right=of PC] (c11) {0} node[n2, draw, right=0 of c11,fill=blue!10] (c12) {3} node[n2, draw, right=0 of c12] (c13) {};
      \draw node[n2, draw, right=of c13] (c21) {1}  node[n2, draw, right=0 of c21,fill=blue!10] (c22) {5} node[n2, draw, right=0 of c22] (c23) {};
      \draw node[n2, draw, right=of c23] (c31) {3}  node[n2, draw, right=0 of c31,fill=blue!10] (c32) {8} node[n2, draw, right=0 of c32] (c33) {$\wedge$};
      \draw[e] (PC.center) -- (c11);
      \draw[e] (c13.center) -- (c21);
      \draw[e] (c23.center) -- (c31);

      \draw node[n2, draw, right=of PD] (c11) {2} node[n2, draw, right=0 of c11,fill=blue!10] (c12) {6} node[n2, draw, right=0 of c12] (c13) {$\wedge$};
      \draw[e] (PD.center) -- (c11);
    \end{tikzpicture}
  }
\end{frame}

\begin{frame}[fragile]
  \frametitle{练习}
  \begin{enumerate}
  \item 请写出数组存储和邻接表的类型定义
  \item 请在如下方面对比数组表示法和邻接表示法
    \begin{itemize}
    \item 存储表示是否唯一
    \item 空间复杂度
    \item 操作a: 求顶点$v_i$的度
    \item 操作b: 判定$(v_i, v_j)$是否是图的一条边
    \item 操作c: 通过遍历求边的数目
    \end{itemize}
  \end{enumerate}
\end{frame}
% TODO

\begin{frame}[fragile]
  \frametitle{比较}
  \small
  \begin{tabular}{| p{2cm} | p{4cm} | p{4cm} |}
    \hline
    ~ & 数组表示法 & 邻接表法 \\ \hline
    表示结果 & 唯一 & 不唯一 \\ \hline
    空间复杂度 & $O(n^2)$ (适用于稠密图) & $O(n+e)$ (适用于稀疏图) \\ \hline
    无向图求顶点$v_i$的度 & 第$i$行(或第$i$列)上非零元素的个数 & 第$i$个边表中的结点个数 \\ \hline
    有向图求顶点$v_i$的度  & 第$i$行上非零元素的个数是$v_i$出度,第$i$列上非零元素的个数是$v_i$的入度 & 第$i$个边表上的结点个数,求入度还需遍历各顶点的边表。逆邻接表则相反\\ \hline
    判定$(v_i, v_j)$是否是图的一条边 &  看矩阵中的$i$行$j$列是否为0 & 扫描第i个边表 \\ \hline
    求边的数目 & 检测整个矩阵中的非零元所耗费的时间是$O(N^2)$ & 对每个边表的结点个数计数所耗费的时间是$O(e+n)$ \\ \hline
  \end{tabular}  
\end{frame}

\begin{frame}[fragile]
  \frametitle{思考}

  怎么把邻接表和逆邻接表相结合,同时表示出来?
\end{frame}

\begin{frame}[fragile]
  \frametitle{有向图的十字链表}
  \begin{columns}[T]
    \column{0.4\textwidth}
    \begin{tikzpicture}[scale=1.2]
      \GraphInit[vstyle=Normal]
      \SetVertexMath
      \begin{scope}[rotate=135]
        \Vertices{circle}{A, C, D, B}
      \end{scope}
      \Edges[style={-Latex}](C,D,A,B)
      \Edges[style={-Latex}](D,B)
      \Edges[style={-Latex, bend left}](C,A,C)
    \end{tikzpicture}

    \column{0.6\textwidth}
    将邻接表、逆邻接表结合起来.
  \end{columns}
  \scalebox{0.7}{
    \begin{tikzpicture}[ n/.style={minimum height=0.8cm, minimum width=1cm},
      n2/.style={minimum height=0.6cm, minimum width=1cm, fill=red!5},
      n3/.style={minimum height=0.6cm, minimum width=1cm},
      e/.style={->, thick, red},
      e2/.style={->, thick, blue}]
      \draw node[n] (n0) {索引} node[n,right=0 of n0](label) {头节点};

      \foreach \x  [evaluate = \x as \xp using int(\x-1)] in {1, ..., 4}
      \draw node[n, below=0 of n\xp](n\x) {$\xp$};

      \foreach \x/\y/\z/\v in {1/A//,2/B//,3/C//,4/D//}
      \draw node[n, right=0 of n\x, draw, fill=yellow!10](\y) {$\y$} node[n, right=0 of \y, draw] (firstIn\y) {\z} node[n, right=0 of firstIn\y, draw] (firstOut\y) {\v};

      \draw node[n2, draw, right=of firstOutA] (headAB) {0} node[n2, draw, right=0 of headAB] (tailAB) {1} node[n3, draw, right=0 of tailAB] (headLinkAB) {} node[n3, draw, right=0 of headLinkAB] (tailLinkAB) {};

      \draw node[n2, draw, right=of tailLinkAB] (headAC) {0} node[n2, draw, right=0 of headAC] (tailAC) {2} node[n3, draw, right=0 of tailAC] (headLinkAC) {} node[n3, draw, right=0 of headLinkAC] (tailLinkAC) {$\wedge$};

      \draw node[n2, draw, right=of firstOutC] (headCA) {2} node[n2, draw, right=0 of headCA] (tailCA) {0} node[n3, draw, right=0 of tailCA] (headLinkCA) {} node[n3, draw, right=0 of headLinkCA] (tailLinkCA) {};

      \draw node[n2, draw, right=of tailLinkCA] (headCD) {2} node[n2, draw, right=0 of headCD] (tailCD) {3} node[n3, draw, right=0 of tailCD] (headLinkCD) {} node[n3, draw, right=0 of headLinkCD] (tailLinkCD) {};

      \draw node[n2, draw, right=of firstOutD] (headDA) {3} node[n2, draw, right=0 of headDA] (tailDA) {0} node[n3, draw, right=0 of tailDA] (headLinkDA) {$\wedge$} node[n3, draw, right=0 of headLinkDA] (tailLinkDA) {};

      \draw node[n2, draw, right=of tailLinkDA] (headDB) {3} node[n2, draw, right=0 of headDB] (tailDB) {1} node[n3, draw, right=0 of tailDB] (headLinkDB) {} node[n3, draw, right=0 of headLinkDB] (tailLinkDB) {};

      \draw[] node[above=1.5cm of firstInA](tipFirstIn) {$firstIn$} node[right=0 of tipFirstIn] (tipFirstOut) {$firstOut$} 
      node[above=1cm of headAB] (tipHeadAB) {$headVertex$} node[above=1.5cm of tailAB] (tipTailAB) {$tailVertex$}
      node[above=1cm of headLinkAB] (tipHeadLinkAB) {$hlink$} node[right=0 of tipHeadLinkAB] (tipTailLinkAB) {$tlink$};

      % draw tip of node
      \path[draw, <-, dashed] (tipFirstIn) edge (firstInA) 
      (tipFirstOut) edge[bend right] (firstOutA)
      (tipHeadAB) edge[] (headAB)
      (tipTailAB) edge[bend left] (tailAB)
      (tipHeadLinkAB) edge[bend left] (headLinkAB)
      (tipTailLinkAB) edge[bend left] (tailLinkAB); 

      % draw links
      \path[e] (firstInA.center) edge[bend left=45, out=120] (headLinkCA.north);
      \draw[e] (headLinkCA.center) -- (headLinkDA.north);
      \draw[e2] (firstOutA.center) -- (headAB);
      \draw[e2] (tailLinkAB.center) -- (headAC);
    \end{tikzpicture}
  }
\end{frame}

\begin{frame}[fragile]
  \frametitle{有向图的十字链表}
  \begin{minted}{java}
    class Vertex {
      String data;
      ArcBox firstIn;
      ArcBox firstOut;
    }

    class ArcBox {
      int headVertex, tailVertex;
      ArcBox hlink;
      ArcBox tlink;
      String data;
    }
  \end{minted}
\end{frame}


\begin{frame}[fragile]
  \frametitle{无向图的多重邻接表}
  \begin{tikzpicture}[scale=1.2]
    \GraphInit[vstyle=Normal]
    \SetVertexMath
    \begin{scope}[rotate=135]
      \Vertices{circle}{A, C, D, B}
    \end{scope}
    \Edges[](A, B, D, C,A,D)
  \end{tikzpicture}


  
\end{frame}
% \section{Search}


\begin{frame}[fragile]{查找表}

  \begin{center}
    \begin{matrixtable}{1.2cm}{4cm}{1.6cm}{0.6cm}{
        \head{Rank}   & \head{Distribution} & \head{Hits} & \\
        1 & Ubuntu    & 2114 & \down  \\
        2 & Fedora    & 1451 & \up    \\
        3 & Mint      & 1297 & \const \\
        4 & OpenSUSE  & 1228 & \up    \\
        5 & Debian    & 910  & \down  \\ }
    \end{matrixtable} 
  \end{center}

  查找是许多应用系统中最消耗时间的一部分,一个好的查找算法会大大提高运行速度。计
  算机需要存储包含该特定信息的表,才可以高效查找。
\end{frame}


\begin{frame}[fragile]{查找表的分类}
  \begin{easylist} \easyitem
    & 静态查找表
    && 仅作查询和检索操作的查找表。
    & 动态查找表
    && 有时在查询之后,还需要将“查询”结果为“不在查找表中”的数据元素{\em 插入}到查找表中;或者,从查找表中{\em 删除}其“查询”结果为“在查找表中”的数据元素。
  \end{easylist}
\end{frame}


\begin{frame}[fragile]{关键字}
  \begin{easylist} \easyitem
    & 是数据元素(或记录)中某个数据项的值,用以标识(识别)一个数据元素(或记录)。
    & 若此关键字可以识别唯一的一个记录,则称之谓“主关键字”。
    & 若此关键字能识别若干记录,则称之谓“次关键字”。
  \end{easylist}
\end{frame}


\begin{frame}[fragile]{查找}
  \begin{easylist} \easyitem
    & 根据给定的某个值,在查找表中确定一个其关键字等于给定值的数据元素或(记录)  
    & 若查找表中存在这样一个记录,则称“查找成功”:
    && 查找结果:给出整个记录的信息,或指示该记录在查找表中的位置;
    & 否则称“查找不成功”,查找结果:
    && 给出“空记录”或“空指针”。
  \end{easylist}
\end{frame}


\begin{frame}[fragile]{如何进行查找?}
  \begin{easylist} \easyitem
    & 查找的方法取决于查找表的结构。
    & 由于查找表中的数据元素之间不存在明显的组织规律,因此不便于查找。
    & 为了提高查找的效率, 需要在查找表中的元素之间人为地 附加某种确定的关系,换句话说, 用另外一种结构来表示查找表。
  \end{easylist}
\end{frame}


\begin{frame}[fragile]{本章大纲}
  \begin{center}
    \smartdiagram[bubble diagram]{查找表, 1. 静态查找表, 2. 动态查找表, 3. 哈希表}
  \end{center}
\end{frame}

\subsection{1. 静态查找表}
\begin{frame}[plain]
  \frametitle{}
  \centering
  \tikzstyle{mybox} = [draw=blue, fill=green!20, very thick,
  rectangle, rounded corners, inner sep=10pt, inner ysep=20pt]
  \tikzstyle{fancytitle} =[fill=blue, text=white, ellipse]
  
  \vspace{1.0cm}
  \begin{tikzpicture}[transform shape, rotate=0, baseline=-3.5cm]
    \node [mybox] (box) {%
      \begin{minipage}[t!]{0.75\textwidth}
        静态查找:对查找集合只进行查找,不涉及插入和删除操作。或者经过一段时间的查找之后,集中地进行插入和删除等修改操作。

        包括:

        \begin{itemize}
        \item 顺序查找
        \item 折半查找
        \item 分块查找
        \end{itemize}
      \end{minipage}
    };
    \node[fancytitle] at (box.north) {1. 静态查找表};
  \end{tikzpicture}
\end{frame}

\begin{frame}[fragile]
  \frametitle{顺序查找}
  \begin{easylist} \easyitem
    & 又称线性查找,是最基本的查找方法之一

    & 从表的一端向另一端逐个按给定值与关键码进行比较,若找到,查找成功,返回数据元素
    在表中的位置;若未找到与$k$相同的关键码,则返回失败信息。

    & 例:查找 $k=35$
  \end{easylist}
  
  \begin{center}
    \begin{tikzpicture}[box/.style={draw, inner sep=0.2cm, minimum size=1cm}]
      \draw[draw] node[box, fill=blue!20] (b0) {~}
      node[box, right=0 of b0] (b1) {10} 
      node[box, right=0 of b1] (b2) {15}
      node[box, right=0 of b2] (b3) {24}
      node[box, right=0 of b3] (b4) {6}
      node[box, right=0 of b4] (b5) {12}
      node[box, right=0 of b5, fill=red!10] (b6) {35}
      node[box, right=0 of b6] (b7) {40}
      node[box, right=0 of b7] (b8) {98}
      node[box, right=0 of b8] (b9) {55}; 

      \foreach \i in {0,...,9}
      {
        \draw node[above=0 of b\i] (idx_\i) {$\i$};
      };

      \path[] (b9.south) ++(0,-0.8cm) edge[-Latex, dashed] node[right]{$i$} (b9.south);
      \path[] (b6.south) ++(0,-0.8cm) edge[-Latex, very thick, draw=red] node[right]{$i$} (b6.south);

      \path[] (b8.south) ++(0,-1.2cm) edge[-Latex, very thick, draw=blue!60] node[above]{查找方向} ++(-6.5cm,0);
    \end{tikzpicture}
  \end{center}

  注意:下标为0的位置,其哨兵用途。
\end{frame}

\begin{frame}[plain]
  % Define box and box title style
  \tikzstyle{mybox} = [draw=red, fill=blue!20, very thick,
  rectangle, rounded corners, inner sep=10pt, inner ysep=20pt]
  \tikzstyle{fancytitle} =[fill=red, text=white]

  \begin{tikzpicture}
    \node [mybox] (box){%
      \begin{minipage}{0.80\textwidth}
        \begin{itemize}
        \item 分析查找算法的效率,通常用平均查找长度ASL (Average Search Length) 来衡
          量,即在查找成功时所进行的关键码比较次数的期望值。

          顺序查找(等概率情况下):

          \[
            ASL = \sum_{i=1}^{n}\dfrac{1}{n}(n-i+1) = \dfrac{n+1}{2}
          \]

          实际上,数据的查找概率存在相当大的差别!

        \item 在查找概率不同的情况下,应遵循查找表需依据查找概率越高,比较次数越少;查找概率
          越低,比较次数就较多的原则来存储数据元素。
        \end{itemize}
      \end{minipage}
    };
    \node[fancytitle, right=10pt] at (box.north west) {顺序查找的性能分析};
    % \node[fancytitle, rounded corners] at (box.east) {$\clubsuit$};
  \end{tikzpicture}
\end{frame}

\begin{frame}[fragile]
  \frametitle{顺序查找总结}
  \begin{easylist} \easyitem
    & 优点:算法简单而且使用面广。
    
    && 对表中记录的存储没有任何要求,顺序存储和链接存储均可(当然, 链式也只能用顺序
    查找);

    && 对表中记录的有序性也没有要求,无论记录是否按关键码有序均可。

    & 缺点:平均查找长度较大,特别是当待查找集合中元素较多时,查找效率较低。   
  \end{easylist}
\end{frame}

\begin{frame}[fragile]
  \frametitle{(有序表)折半查找}
  \begin{easylist} \easyitem
    & 有序表是表中数据元素按关键码升序或降序排列。

    & 适用于:

    && 线性表中的记录必须按关键码有序;
    
    && 必须采用顺序存储。
  \end{easylist}
\end{frame}

\begin{frame}[fragile]
  \frametitle{请查找14}

  \scalebox{0.7}{
    \begin{tikzpicture}[box/.style={draw, inner sep=0.2cm, minimum size=1cm}]
      \draw[draw] node[box] (b0) {~}
      node[box, right=0 of b0] (b1) {7} 
      node[box, right=0 of b1] (b2) {14}
      node[box, right=0 of b2] (b3) {18}
      node[box, right=0 of b3] (b4) {21}
      node[box, right=0 of b4] (b5) {23}
      node[box, right=0 of b5] (b6) {29}
      node[box, right=0 of b6] (b7) {31}
      node[box, right=0 of b7] (b8) {35}
      node[box, right=0 of b8] (b9) {38}
      node[box, right=0 of b9] (b10) {42}
      node[box, right=0 of b10] (b11) {46}
      node[box, right=0 of b11] (b12) {49}
      node[box, right=0 of b12] (b13) {52} ; 

      \foreach \i in {0,...,13}
      {
        \draw node[above=0 of b\i] (idx_\i) {\i};
      };

      \path[] (b1.south) ++(0,-0.5cm) edge[-Latex, thick] node[below left]{$low=1$} (b1.south);
      \path[] (b13.south) ++(0,-0.5cm) edge[-Latex, thick] node[below right]{$high=13$} (b13.south);
      \path[] (b0.south) ++(-0.5cm,-1cm) edge[draw, dashed] node[above]{\textcircled{1} 设置初始区间} ++(14.5cm,0);


      \path[] (b7.south) ++(0,-1.6cm) edge[-Latex, thick] node[below]{$mid=7$} ++(0,0.5cm)  node[right, xshift=1cm]{\textcircled{2}调整到左半区};

      \path[] (b1.south) ++(0,-2.5cm) edge[-Latex, thick] node[below left]{$low=1$} ++(0,0.5cm);
      \path[] (b6.south) ++(0,-2.5cm) edge[-Latex, thick] node[below right]{$high=6$} ++(0,0.5cm);
      \path[] (b0.south) ++(-0.5cm,-3cm) edge[draw, dashed] node[above]{} ++(14.5cm,0);


      \path[] (b3.south) ++(0,-4cm) edge[-Latex, thick] node[below]{$mid=3$} ++(0,0.5cm)  node[right, xshift=1cm]{\textcircled{3}调整到左半区};

      \path[] (b1.south) ++(0,-4.8cm) edge[-Latex, thick] node[below left]{$low=1$} ++(0,0.5cm);
      \path[] (b2.south) ++(0,-4.8cm) edge[-Latex, thick] node[below right]{$high=2$} ++(0,0.5cm);
      \path[] (b0.south) ++(-0.5cm,-5.2cm) edge[draw, dashed] node[above]{} ++(14.5cm,0);


      \path[] (b1.south) ++(0,-6cm) edge[-Latex, thick] node[below]{$mid=1$} ++(0,0.5cm) node[right, xshift=1cm]{\textcircled{4}调整到右半区};

      \path[] (b2.south) ++(0,-6.8cm) edge[-Latex, thick] node[below left]{$low=2$} ++(0,0.5cm);
      \path[] (b2.south) ++(0,-6.8cm) edge[-Latex, thick] node[below right]{$high=2$} ++(0,0.5cm);
      \path[] (b0.south) ++(-0.5cm,-7.2cm) edge[draw, dashed] node[above]{} ++(14.5cm,0);

      
      \path[] (b2.south) ++(0,-8cm) edge[-Latex, thick] node[below]{$mid=2$} ++(0,0.5cm) node[right, xshift=1cm]{\textcircled{5}查找成功};
    \end{tikzpicture}
  } 
\end{frame}

\begin{frame}[fragile]
  \frametitle{课堂练习:请查找22}
  
  \scalebox{0.7}{
    \begin{tikzpicture}[box/.style={draw, inner sep=0.2cm, minimum size=1cm}]
      \draw[draw] node[box] (b0) {~}
      node[box, right=0 of b0] (b1) {7} 
      node[box, right=0 of b1] (b2) {14}
      node[box, right=0 of b2] (b3) {18}
      node[box, right=0 of b3] (b4) {21}
      node[box, right=0 of b4] (b5) {23}
      node[box, right=0 of b5] (b6) {29}
      node[box, right=0 of b6] (b7) {31}
      node[box, right=0 of b7] (b8) {35}
      node[box, right=0 of b8] (b9) {38}
      node[box, right=0 of b9] (b10) {42}
      node[box, right=0 of b10] (b11) {46}
      node[box, right=0 of b11] (b12) {49}
      node[box, right=0 of b12] (b13) {52} ; 

      \foreach \i in {0,...,13}
      {
        \draw node[above=0 of b\i] (idx_\i) {\i};
      };
    \end{tikzpicture}
  } 
\end{frame}

\begin{frame}[fragile]
  \frametitle{}
  \begin{tikzpicture}[level distance=10mm]
    \tikzstyle{every node}=[draw,circle,inner sep=3pt, minimum size=0.5cm]
    \tikzstyle{level 1}=[sibling distance=60mm,
    set style={{every node}+=[]}]
    \tikzstyle{level 2}=[sibling distance=30mm,
    set style={{every node}+=[]}]
    \tikzstyle{level 3}=[sibling distance=20mm,
    set style={{every node}+=[]}]
    \node {31}
    child {node {18}
      child {node {7}         
        child[right] {node {14}}
      }
      child {node {23}
        child {node {21}}
        child {node {29}}
      }
    }
    child {node {42}
      child {node {35}
        child[right] {node {38}}
      }
      child {node {49}
        child {node {46}}
        child {node {52}}
      }
    };
  \end{tikzpicture}

  从折半查找过程看,以表的中点为比较对象,并以中点将表分割为两个子表,对定位到的子表
  继续这种操作。所以,对表中每个数据元素的查找过程,可用二叉树来描述。
  
  \begin{easylist} \easyitem
    & 折半查找在查找成功时,所进行的关键码比较次数至多为?
    
    & 请问平均查找长度(ASL)是多少?
  \end{easylist}
\end{frame}

\begin{frame}[fragile]
  \frametitle{}

  \begin{tikzpicture}[level distance=10mm]
    \tikzstyle{every node}=[draw,circle,inner sep=3pt, minimum size=0.5cm]
    \tikzstyle{level 1}=[sibling distance=60mm,
    set style={{every node}+=[]}]
    \tikzstyle{level 2}=[sibling distance=30mm,
    set style={{every node}+=[]}]
    \tikzstyle{level 3}=[sibling distance=20mm,
    set style={{every node}+=[]}]
    \node {31}
    child {node {18}
      child {node {7}         
        child[right] {node {14}}
      }
      child {node {23}
        child {node {21}}
        child {node {29}}
      }
    }
    child {node {42}
      child {node {35}
        child[right] {node {38}}
      }
      child {node {49}
        child {node {46}}
        child {node {52}}
      }
    };
  \end{tikzpicture}

  \begin{easylist} \easyitem
    & 折半查找在查找成功时,所进行的关键码比较次数至多为?

    \[
      \lfloor log_2 n \rfloor + 1
    \]
    
    & 请问平均查找长度(ASL)是多少?

    \[
      ASL = \dfrac{1}{n}[1 \times 2^0 + 2 \times 2^1 + \cdots + k \times  2^{k-1}]
      \approx \dfrac{n+1}{n} log_2(n+1) - 1
    \]   
  \end{easylist}
\end{frame}


\begin{frame}[fragile]
  \frametitle{分块查找}
  \begin{easylist} \easyitem

    & 分块查找又称索引顺序查找,是对顺序查找的一种改进。适用于表有序或者分块有
    序(后面的子表中所有记录的关键码均大于前一个子表的最大关键码)的情形。

    & 例:对某集合按关键码值31,62,88分为三块建立的查找表及其索引表如下:
  \end{easylist}
  
  \begin{center} 
  \scalebox{0.8}{
    \begin{tikzpicture}[box/.style={draw, minimum size=0.6cm},box2/.style={draw, minimum size=0.7cm, minimum width=1cm, fill=yellow!10}]
      \draw[draw] node[box] (b1) {14} 
      node[box, right=0 of b1] (b2) {31}
      node[box, right=0 of b2] (b3) {8}
      node[box, right=0 of b3] (b4) {22}
      node[box, right=0 of b4] (b5) {18}
      node[box, right=0 of b5] (b6) {43}
      node[box, right=0 of b6] (b7) {62}
      node[box, right=0 of b7] (b8) {49}
      node[box, right=0 of b8] (b9) {35}
      node[box, right=0 of b9] (b10) {52}
      node[box, right=0 of b10] (b11) {88}
      node[box, right=0 of b11] (b12) {78}
      node[box, right=0 of b12] (b13) {70} 
      node[box, right=0 of b13] (b14) {82}
      node[above left=0.2 of b1] {查找表}; 

      \foreach \i in {1,...,14}
      {
        \draw node[below=0 of b\i] (idx_\i) {\i};
      };

      \draw[draw] node[box2, above=1.5cm of b5] (x1) {31} 
      node[box2, right=0 of x1] (x2) {62}
      node[box2, right=0 of x2] (x3) {88}
      node[box2, below=0 of x1] (y1) {1}
      node[box2, below=0 of x2] (y2) {6}
      node[box2, below=0 of x3] (y3) {11}
      node[left=of x1] {索引表}
      node[right=0.2 of x3] {关键码字段}
      node[right=0.2 of y3] {指针字段};

      \path[draw] (y1.center) edge[-Latex] (b1.north) (y2.center) edge[-Latex] (b6.north) (y3.center) edge[-Latex] (b11.north);
    \end{tikzpicture}
  }
\end{center}
 
\end{frame}

\begin{frame}[fragile]
  \frametitle{分块查找}
  \begin{easylist} \easyitem

    & 分块查找要求将查找表分成若干个子表,并对子表建立索引表,查找表的每一个子表由
    索引表中的索引项确定。
    
    & 索引项

    && 关键码字段 (存放对应子表中的最大关键码值) ;
    && 指针字段 (存放指向对应子表的指针) ,并且要求索引项按关键码字段有序。
    
    & 如何根据索引表和查找表进行查找?
  \end{easylist}
  
  \begin{center} 
  \scalebox{0.8}{
    \begin{tikzpicture}[box/.style={draw, minimum size=0.6cm},box2/.style={draw, minimum size=0.7cm, minimum width=1cm, fill=yellow!10}]
      \draw[draw] node[box] (b1) {14} 
      node[box, right=0 of b1] (b2) {31}
      node[box, right=0 of b2] (b3) {8}
      node[box, right=0 of b3] (b4) {22}
      node[box, right=0 of b4] (b5) {18}
      node[box, right=0 of b5] (b6) {43}
      node[box, right=0 of b6] (b7) {62}
      node[box, right=0 of b7] (b8) {49}
      node[box, right=0 of b8] (b9) {35}
      node[box, right=0 of b9] (b10) {52}
      node[box, right=0 of b10] (b11) {88}
      node[box, right=0 of b11] (b12) {78}
      node[box, right=0 of b12] (b13) {70} 
      node[box, right=0 of b13] (b14) {82}
      node[above left=0.2 of b1] {查找表}; 

      \foreach \i in {1,...,14}
      {
        \draw node[below=0 of b\i] (idx_\i) {\i};
      };

      \draw[draw] node[box2, above=1.5cm of b5] (x1) {31} 
      node[box2, right=0 of x1] (x2) {62}
      node[box2, right=0 of x2] (x3) {88}
      node[box2, below=0 of x1] (y1) {1}
      node[box2, below=0 of x2] (y2) {6}
      node[box2, below=0 of x3] (y3) {11}
      node[left=of x1] {索引表}
      node[right=0.2 of x3] {关键码字段}
      node[right=0.2 of y3] {指针字段};

      \path[draw] (y1.center) edge[-Latex] (b1.north) (y2.center) edge[-Latex] (b6.north) (y3.center) edge[-Latex] (b11.north);
    \end{tikzpicture}
  }
\end{center}

\end{frame}

\begin{frame}[fragile]
  \frametitle{分块查找性能分析}

  \begin{easylist}
    & 分块查找含索引表查找和子表查找。
    
    & 设$n$个数据元素的查找表分为$m$个相同大小的子表。则分块查找的平均查找长度为:

    \[
      ASL=\dfrac{m+1}{2} + \dfrac{1}{2} \cdot \dfrac{n}{m+1}
      = \dfrac{m+\dfrac{n}{m}}{2}+1
    \]

    & 可见,平均查找长度和表的总长度$n$、子表个数$m$有关。
  \end{easylist}
  \begin{center} 
  \scalebox{0.8}{
    \begin{tikzpicture}[box/.style={draw, minimum size=0.6cm},box2/.style={draw, minimum size=0.7cm, minimum width=1cm, fill=yellow!10}]
      \draw[draw] node[box] (b1) {14} 
      node[box, right=0 of b1] (b2) {31}
      node[box, right=0 of b2] (b3) {8}
      node[box, right=0 of b3] (b4) {22}
      node[box, right=0 of b4] (b5) {18}
      node[box, right=0 of b5] (b6) {43}
      node[box, right=0 of b6] (b7) {62}
      node[box, right=0 of b7] (b8) {49}
      node[box, right=0 of b8] (b9) {35}
      node[box, right=0 of b9] (b10) {52}
      node[box, right=0 of b10] (b11) {88}
      node[box, right=0 of b11] (b12) {78}
      node[box, right=0 of b12] (b13) {70} 
      node[box, right=0 of b13] (b14) {82}
      node[above left=0.2 of b1] {查找表}; 

      \foreach \i in {1,...,14}
      {
        \draw node[below=0 of b\i] (idx_\i) {\i};
      };

      \draw[draw] node[box2, above=1.5cm of b5] (x1) {31} 
      node[box2, right=0 of x1] (x2) {62}
      node[box2, right=0 of x2] (x3) {88}
      node[box2, below=0 of x1] (y1) {1}
      node[box2, below=0 of x2] (y2) {6}
      node[box2, below=0 of x3] (y3) {11}
      node[left=of x1] {索引表}
      node[right=0.2 of x3] {关键码字段}
      node[right=0.2 of y3] {指针字段};

      \path[draw] (y1.center) edge[-Latex] (b1.north) (y2.center) edge[-Latex] (b6.north) (y3.center) edge[-Latex] (b11.north);
    \end{tikzpicture}
  }
\end{center}

\end{frame}


\subsection{2. 动态查找表}
\begin{frame}[plain]
  \frametitle{}
  \centering
  \tikzstyle{mybox} = [draw=blue, fill=green!20, very thick,
  rectangle, rounded corners, inner sep=10pt, inner ysep=20pt]
  \tikzstyle{fancytitle} =[fill=blue, text=white, ellipse]
  
  \vspace{1.0cm}
  \begin{tikzpicture}[transform shape, rotate=0, baseline=-3.5cm]
    \node [mybox] (box) {%
      \begin{minipage}[t!]{0.75\textwidth}
        动态查找表的特点是,表结构本身是在查找过程中动态生成的,即对于给定的key,若
        表中存在其关键字等于key的记录,则查找成功返回,否则插入关键字等于key的记
        录。

        包括:

        \begin{itemize}
        \item 二叉排序树
        \item 平衡二叉树
        \end{itemize}
      \end{minipage}
    };
    \node[fancytitle] at (box.north) {2. 动态查找表};
  \end{tikzpicture}
\end{frame}

\begin{frame}[fragile]
  \frametitle{二叉排序树}
  \begin{columns}[T] % align columns
    \begin{column}{0.58\linewidth}
      \begin{itemize}
      \item 二叉排序树(Binary Sort Tree)或者是一棵空树;或者是具有下列性质的二叉树:

        \textcircled{1} 若左子树不空,则左子树上所有结点的值均小于根结点的值;若
        右子树不空,则右子树上所有结点的值均大于根结点的值。

        \textcircled{2} 左右子树也都是二叉排序树。

      \item 对二叉排序树进行中序遍历,可以得到一个按关键码有序的序列,因此,一个无序序列
        可通过构造二叉排序树而成为有序序列。
      \end{itemize}
    \end{column}
    \hfill
    \begin{column}{0.38\linewidth}
      \scalebox{0.7}{
        \begin{forest}
          [ 63
          [55
          [42  [10]    [45]  ]
          [58]
          ]
          [90
          [70 [67] [83]]
          [98]
          ]
          ]
        \end{forest}
      }
    \end{column}
  \end{columns}
\end{frame}

\begin{frame}[fragile]
  \frametitle{二叉排序树的查找}

  \begin{columns}[T] % align columns
    \begin{column}{0.58\linewidth}
      \begin{itemize}
      \item 若查找树为空,查找失败;否则将key与查找树的根结点比较

        \textcircled{1} 若相等,查找成功,否则,

        \textcircled{2} 如果key<根结点关键码,继续在以左子树上进行查找

        \textcircled{3} 如果key>根结点关键码,继续在以右子树上进行查找

      \item 例如在右图所示的树上查找45
      \end{itemize}
    \end{column}
    \hfill
    \begin{column}{0.38\linewidth}
      \scalebox{0.7}{
        \begin{forest}
          [ 63
          [55, edge={->, draw=red, thick}
          [42, edge={->, draw=red, thick}  [10]    [45, edge={->, draw=red, thick}]  ]
          [58]
          ]
          [90
          [70 [67] [83]]
          [98]
          ]
          ]
        \end{forest}
      }
    \end{column}
  \end{columns} 
\end{frame}

\begin{frame}[fragile]
  \frametitle{二叉排序树的查找(cont.)}

  \begin{minipage}{0.6\textwidth}
    \begin{easylist}
      & 两树的平均查找长度分别为:

      \[
        ASL_a = \dfrac{1}{6} \times [1+2+2+3+3+3] = \dfrac{14}{6}
      \]
      
      \[
        ASL_b = \dfrac{1}{6} \times [1+2+3+4+5+6] = \dfrac{21}{6}
      \]
      
      & 二叉排序树的平均查找长度和树的形态有关!最好情况是$O(log_2 n)$.
    \end{easylist}
  \end{minipage}%
  \begin{minipage}{0.36\textwidth}
    \scalebox{0.6} {
      \begin{forest}
        [12, grow=-45 [24, grow=-45 [37, grow=-45 [45, grow=-45 [53,grow=-60
        [93]]]]]]         
      \end{forest}
    }
    \scalebox{0.6}{
      \begin{forest}
        [45 [24 [12] [37]] [53,grow=-60 [93]]]
      \end{forest}
    }
  \end{minipage}
\end{frame}

\begin{frame}[fragile]
  \frametitle{二叉排序树的构建 --- 插入节点}
  \begin{minipage}{0.6\textwidth}
    \begin{itemize}
    \item 在查找不成功时,插入该key
      \begin{itemize}
      \item 新插入结点一定是作为叶子结点添加的
      \item 插入位置在查找过程中得到
      \end{itemize}
    \item 例如查找56
    \end{itemize}
  \end{minipage}%
  \begin{minipage}{0.36\textwidth}    
    \scalebox{0.6}{
      \begin{forest}
        [63 [55 [42 [10] [ 45]] [58, grow=245 [56, fill=red!50, dotted]]] [90 [70 [67] [83]] [98]]]
      \end{forest}
    }
  \end{minipage}
\end{frame}

\begin{frame}[fragile]
  \frametitle{序列: 63, 90, 70, 55, 67, 42, 98, 83, 10, 45, 58}
  \small
  \scalebox{0.65}{
    \begin{forest}
      [63]
    \end{forest}\quad
    \begin{forest}
      [63 [, missed] [90, fill=red!10]]
    \end{forest}\quad
    \begin{forest}
      [63 [, missed] [90 [70, fill=red!10] [, missed]]]
    \end{forest}\quad
    \begin{forest}
      [63 [55, fill=red!10] [90 [70] [, missed]]]
    \end{forest}\quad
    \begin{forest}
      [63 [55] [90 [70 [67,fill=red!10] [, missed]] [, missed]]]
    \end{forest}\quad
    \begin{forest}
      [63 [55 [42,fill=red!10] [, missed]] [90 [70 [67] [, missed]] [, missed]]]
    \end{forest}\quad
    \begin{forest} 
      [63 [55 [42] [, missed]] [90 [70 [67] [, missed]] [98, fill=red!10]]]
    \end{forest}
  }

  \scalebox{0.6}{
    
    \begin{forest} 
      [63 [55 [42] [, missed]] [90 [70 [67] [83, fill=red!10]] [98]]]
    \end{forest}
    \quad
    \begin{forest} 
      [63 [55 [42 [10, fill=red!10] [,missed]] [, missed]] [90 [70 [67] [83]] [98]]]
    \end{forest}
    \quad
    \begin{forest} 
      [63 [55 [42 [10] [45, fill=red!10]] [, missed]] [90 [70 [67] [83]] [98]]]
    \end{forest}
    \quad
    \begin{forest} 
      [63 [55 [42 [10] [45]] [58, fill=red!10]] [90 [70 [67] [83]] [98]]]
    \end{forest}
  }
\end{frame}

\begin{frame}[fragile]
  \frametitle{二叉排序树的删除操作}

  依次删除结点45、90,仍要使树保持二叉排序树的特性

  
\end{frame}

\begin{frame}[fragile]{}
  \begin{easylist} \easyitem

  \end{easylist}
\end{frame}



% \section{Sorting}


\begin{frame}[plain]
  \begin{outlinebox}{内部排序大纲}
    \begin{itemize}
    \item 排序的基本概念
    \item 具体排序方法
      \begin{enumerate}
      \item \color{red} 插入排序:直接插入排序
      \item 插入排序:折半插入排序
      \item 插入排序:希尔排序
        
      \item \color{blue} 交换排序:冒泡法
      \item 交换排序:快速排序
        
      \item \color{orange} 选择排序:简单选择排序
      \item 选择排序:堆排序
        
      \item \color{purple} 归并排序:二路归并
      \item \color{gray} 基数排序
      \end{enumerate}
    \end{itemize}    
  \end{outlinebox}
\end{frame}

\begin{frame}[fragile]
  \frametitle{排序}
  \begin{easylist} \easyitem

    & 对一个数据元素集合或序列重新排列成一个按数据元素某个项值有序的序列就是排
    序。

    && 例如将关键字序列:

    $52, 49, 80, 36, 14, 58, 61, 23, 97, 75$

    调整为
    
    $14, 23, 36, 49, 52, 58, 61 ,75, 80, 97$

    && 再如将:

    $<Susie,26>, <Jack,22>, <Michel,25>, <Richard,25>$

    调整为:

    $<Jack,22>,<Michel,25>, <Richard, 25>, <Susie,26>$
  \end{easylist}
\end{frame}

\begin{frame}[fragile]
  \frametitle{排序的稳定性}
  \begin{easylist} \easyitem

    & 请注意刚才第二个序列的排序结果不唯一!

    $<Susie,26>, <Jack,22>, <Michel,25>, <Richard,25>$

    \color{red} $<Jack,22>,<Michel,25>, <Richard, 25>, <Susie,26>$

    \color{blue} $<Jack,22>, <Richard, 25>, <Michel,25>, <Susie,26>$
    
    & 稳定:若存在相同的关键字,对应位置的记录在排序后仍然保持原来的顺序,则称所使用
    的排序方法是稳定的。反之称为不稳定的。

  \end{easylist}
\end{frame}

\begin{frame}[fragile]
  \frametitle{}
  \begin{sectionbox}{插入排序}
    \begin{itemize}
    \item 直接插入排序
    \item 折半插入排序
    \item 希尔排序
    \end{itemize}
  \end{sectionbox}
\end{frame}

\begin{frame}[fragile]
  \frametitle{直接插入排序}

  对于要插入的元素$R[i]$,从$R[i-1]$起向前进行顺序查找,当$R[j-1]$小于$R[i]$时停止,插
  入位置为$R[j]$。注意在顺序表中要移动元素实现元素的插入。
  
\end{frame}


\end{document}


%%% Local Variables:
%%% mode: latex
%%% TeX-master: t
%%% End:
