\section{内部排序}

\begin{frame}[plain]
  \begin{outlinebox}{内部排序大纲}
    \begin{itemize}
    \item 排序的基本概念
    \item 具体排序方法
      \begin{enumerate}
      \item \color{red} 插入排序:直接插入排序
      \item 插入排序:折半插入排序
      \item 插入排序:希尔排序
        
      \item \color{blue} 交换排序:冒泡法
      \item 交换排序:快速排序
        
      \item \color{orange} 选择排序:简单选择排序
      \item 选择排序:堆排序
        
      \item \color{purple} 归并排序:二路归并
      \item \color{gray} 基数排序
      \end{enumerate}
    \end{itemize}    
  \end{outlinebox}
\end{frame}

\subsection{排序的基本概念}
\begin{frame}[fragile]
  \frametitle{排序}
  \begin{easylist} \easyitem

    & 对一个数据元素集合或序列重新排列成一个按数据元素某个项值有序的序列就是排
    序。

    && 例如将关键字序列:

    $52, 49, 80, 36, 14, 58, 61, 23, 97, 75$

    调整为
    
    $14, 23, 36, 49, 52, 58, 61 ,75, 80, 97$

    && 再如将:

    $<Susie,26>, <Jack,22>, <Michel,25>, <Richard,25>$

    调整为:

    $<Jack,22>,<Michel,25>, <Richard, 25>, <Susie,26>$
  \end{easylist}
\end{frame}

\begin{frame}[fragile]
  \frametitle{排序的稳定性}
  \begin{easylist} \easyitem

    & 请注意刚才第二个序列的排序结果不唯一!

    $<Susie,26>, <Jack,22>, <Michel,25>, <Richard,25>$

    \color{red} $<Jack,22>,<Michel,25>, <Richard, 25>, <Susie,26>$

    \color{blue} $<Jack,22>, <Richard, 25>, <Michel,25>, <Susie,26>$
    
    & 稳定:若存在相同的关键字,对应位置的记录在排序后仍然保持原来的顺序,则称所使用
    的排序方法是稳定的。反之称为不稳定的。

  \end{easylist}
\end{frame}

\subsection{插入排序}
\begin{frame}[fragile]
  \frametitle{}
  \begin{sectionbox}{插入排序}
    \begin{itemize}
    \item 直接插入排序
    \item 折半插入排序
    \item 希尔排序
    \end{itemize}
  \end{sectionbox}
\end{frame}

\subsubsection{直接插入排序}
\begin{frame}[fragile]
  \frametitle{直接插入排序}

  对于要插入的元素$R[i]$,从$R[i-1]$起向前进行顺序查找,当$R[j-1]$小于$R[i]$时停止,插
  入位置为$R[j]$。注意在顺序表中要移动元素实现元素的插入。

  \begin{center}
    \begin{tikzpicture}[box/.style={draw, fill=green!20, minimum size=1cm}]
      \draw[draw] node[box, fill=red!20] (b0) {76}
      node[box, right=0 of b0] (b1) {38} 
      node[box, right=0 of b1] (b2) {49}
      node[box, right=0 of b2] (b3) {65}
      node[box, right=0 of b3] (b4) {97}
      node[box, right=0 of b4, fill=blue!10] (b5) {76}
      node[box, right=0 of b5, fill=blue!10] (b6) {13}
      node[box, right=0 of b6, minimum width=2cm, fill=blue!10] (b7) {$\cdots$}; 

      \foreach \i in {0,...,6}
      {
        \draw node[above=0 of b\i] (idx_\i) {$\i$};
      };

      \draw node[above=0 of idx_5] {$R[i]$} node[above=0 of idx_0] {$R[0]$};

      \path[] (b0.south) ++(0,-1.5cm) edge[-Latex, dashed] node[below, align=center]{设置“哨兵”存储$R[i]$,\\只要待比较的元素大于$R[0]$,\\就继续往前比较,从而实现\\$R[j] \cdots R[i-1]$的后移。} (b0.south);

      \path[] (b4.south) ++(0,-1.5cm) edge[-Latex, very thick, draw=red] node[below right,align=center]{$j$, 指示插入的位置,\\$R[j-1] \leq R[i]$  \\ (稳定的排序方法)} (b4.south);
    \end{tikzpicture}
  \end{center} 
\end{frame}

\begin{frame}[fragile]
  \frametitle{直接插入排序举例}

  \begin{center}
    \begin{tabular}{|c|c|c|c|c|c|c|}
      \hline
      47 & 38 & 65 & 97 & 13 & 27 & \circled{47} \\ \hline
      \cellcolor{blue!20} 47 & \cellcolor{red!20} 38 & 65 & 97 & 13 & 27 & \circled{47} \\ \hline
      \cellcolor{blue!20} 38 &  \cellcolor{blue!20} 47 & \cellcolor{red!20} 65 & 97 & 13 & 27 & \circled{47} \\ \hline
      \cellcolor{blue!20} 38 &  \cellcolor{blue!20} 47 &  \cellcolor{blue!20} 65 & \cellcolor{red!20} 97 & 13 & 27 & \circled{47} \\ \hline
      \cellcolor{blue!20} 38 &  \cellcolor{blue!20} 47 &  \cellcolor{blue!20} 65 & \cellcolor{blue!20} 97 & \cellcolor{red!20} 13 & 27 & \circled{47} \\ \hline
      \cellcolor{blue!20} 13 &  \cellcolor{blue!20} 38 &  \cellcolor{blue!20} 47 & \cellcolor{blue!20} 65 & \cellcolor{blue!20} 97 & \cellcolor{red!20} 27 & \circled{47} \\ \hline
      \cellcolor{blue!20} 13 &  \cellcolor{blue!20} 27 &  \cellcolor{blue!20} 38 & \cellcolor{blue!20} 47 & \cellcolor{blue!20} 65 & \cellcolor{blue!20} 97 & \cellcolor{red!20} \circled{47} \\ \hline
      \cellcolor{blue!20} 13 &  \cellcolor{blue!20} 27 &  \cellcolor{blue!20} 38 & \cellcolor{blue!20} 47 & \cellcolor{blue!20} \circled{47} & \cellcolor{blue!20} 65 & \cellcolor{blue!20} 97 \\ \hline
    \end{tabular}
  \end{center}

  注意:两个47的位置。
\end{frame}

\begin{frame}[fragile]
  \frametitle{直接插入排序算法分析}
  \begin{easylist} \easyitem
    & 空间效率: 用一个辅助单元

    & 时间效率: 时间复杂度为$O(n^2)$
    
    && 进行了n-1次向有序表插入记录的操作,每趟是“比较+移动”

    && 最好情况:记录按关键字正序

    &&& $n-1$次比较,$0$次移动

    && 最差情况:记录按关键字逆序
    
    &&& 比较次数:
    $\sum_{i=2}^n i = 2+3+\cdots + n = \dfrac{(n+2) \times (n-1)}{2}$
    
    &&& 移动次数:
    $\sum_{i=2}^n (i+1) = \dfrac{(n+4) \times (n-1)}{2}$

    && 平均情况:不妨取上述各值的平均,可知比较和移动次数约$n^2/4$

    & 是稳定的排序方法
  \end{easylist}
\end{frame}


\subsubsection{希尔排序}
\begin{frame}[fragile]
  \frametitle{希尔(Shell)排序}
  \begin{infobox}{希尔排序的思想}
    对于元素数量较少,或者基本有序的待排序序列,直接插入排序的效率不错。
   \end{infobox}

  \begin{easylist} \easyitem
    & 根据增量$d$分割出子序列
    
    & 对子序列进行直接插入排序

    & 增量$d$的选择

    && Shell最初的方案:
    $d = n/2, d=d/2, \cdots , d=1$

    && Knuth的方案:
    $d=d/3 + 1$

    && 其它:d为奇数; d互质 $\cdots$
  \end{easylist}
\end{frame}

\begin{frame}[fragile, allowframebreaks]
  \frametitle{希尔排序举例}
  \begin{infobox}{第1趟:$d_1=5$}
    \begin{center}
      \begin{tabular}{|c|c|c|c|c|c|c|c|c|c|}
        \hline
        \rowcolor{yellow!50}
        \circled[red]{49} & 38 & 65  & 97 & 76 & 13 & 27 & \circled{49} & 55 & 04 \\ \hline
        13 & ~   &    &    &    & \circled[red]{49} &    &    &    &     \\ \hline
        ~ & 27   &    &    &    &    & 38 &    &    &     \\ \hline
        ~ &      & \circled{49} &    &    &    &    & 65 &    &     \\ \hline
        ~ &      &    & 55 &    &    &    &    & 97 &     \\ \hline
        ~ &      &    &    & \cellcolor{red!5} 04 &    &    &    &    & 76  \\ \hline
      \end{tabular}
    \end{center}
  \end{infobox}

  \begin{easylist} \easyitem
    & 观察04: 跳跃式的往前移
    & 观察49: 希尔排序不稳定
  \end{easylist}

  \newpage

  \begin{infobox}{第2趟:$d_2=2$}
    \begin{center}
      \begin{tabular}{|c|c|c|c|c|c|c|c|c|c|}
        \hline
        \rowcolor{yellow!50}
        13 & 27  &  \circled{49}  & 55   & 04 & \circled[red]{49}  &  38  &  65  &  97  & 76  \\ \hline
        04 &     &  13  &    &  38 &      & \circled{49} &    &  97  &      \\ \hline
        ~ &  27   &    & \circled[red]{49}   &    &  55  &  &  65  &    & 76    \\ \hline
      \end{tabular}
    \end{center}
  \end{infobox}

  
  \begin{infobox}{第3趟:$d_3=1$}
    \begin{center}
      \begin{tabular}{|c|c|c|c|c|c|c|c|c|c|}
        \hline
        \rowcolor{yellow!50}
        04 & 27  & 13 &  \circled[red]{49}  & 38 & 55 & \circled{49}  &  65  &  97  & 76  \\ \hline
        04 & 13  & 27 &  38 & \circled[red]{49}  & \circled{49} & 55 &  65  &  76  & 97  \\ \hline
      \end{tabular}
    \end{center}
  \end{infobox}
\end{frame}

\begin{frame}[fragile]
  \frametitle{希尔排序分析}
  \begin{easylist} \easyitem
    & shell排序的时间性能优于直接插入排序!
    
    & shell排序开始时增量$d$较大(这使得分组较多,每组记录少),故各组内直接插入较快,后
    来增量$d$渐小(各组记录渐多),但组内元素已经比较接近有序状态,所以新的一趟排序过
    程也比较快。

    & 希尔排序的复杂度分析很复杂

    && 在特定情况下可以准确估算比较、移动次数,但是考虑与增量之间的依赖关系,并给出
    完整的数学分析,目前还做不到

    && 在增量序列为$\delta[k]=2^{t-k+1}$时, 希尔排序的时间复杂度为$O(n^{3/2})$
    
    && Knuth的统计结论是,平均比较次数和对象平均移动次数在$n^{1.25}$与$1.6 \cdot
    n^{1.25}$之间。    
  \end{easylist}
\end{frame}

\begin{frame}[plain]
  \vspace{2cm}
  \begin{infobox}{希尔排序增量序列的取法}
    目前尚未有工作求得一种最好的增量序列。但需注意的是:应使增量序列中的值没有
    除1之外的公因子,并且最后一个增量必须为1。
  \end{infobox}
\end{frame}

