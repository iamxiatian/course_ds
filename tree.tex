\section{Tree}


\begin{frame}[fragile]{树和二叉树}
  树型结构是结点之间有分支,并且具有层次关系的结构,类似于自然界中的树。树有很多应
  用,比如Unix等操作系统中的目录结构。
\end{frame}

\begin{frame}[fragile]
  \frametitle{例子}
\begin{forest}
 [CEO, for tree={rectangle, minimum width=2cm}, fill=red!10
    [CFO [财务人员] ]
    [CTO [工程师] ]
    [CMO [销售人员] ]
 ]
 \node at (current bounding box.south)
 [below=1ex,draw,cloud,aspect=6,cloud puffs=30]
 {\emph{Simple Company Hierarchy}};
\end{forest}
\end{frame}

\begin{frame}[fragile, plain]
  \scalebox{0.7}{
    \begin{forest}
      [学院, for tree={draw=none, rectangle, minimum width=1cm}, fill=red!10, circle
       [社会学部, grow=west [信息资源管理学院, fill=red!10] [新闻学院] [农业与农村发展学院] [社会与人口学院]
       [公共管理学院] [教育学院]]
       [$\cdots$]
       [ 人文学部,grow=east [哲学院] [文学院] [历史学院] [国学院] [艺术学院] [外国语学院] [清史研究所]]
       ]
       \node at (current bounding box.south)
       [below=1ex,draw,cloud,aspect=6,cloud puffs=30]
       {\emph{人民大学学院设置}};
    \end{forest}
  }
\end{frame}

\begin{frame}[fragile]{内容}
  \begin{easylist} \easyitem
    & 树的基本术语
    & 二叉树
    & 遍历二叉树与线索二叉树
    & 树和森林
    & 哈夫曼树
  \end{easylist}
\end{frame}

\subsection{基本术语}

\begin{frame}[fragile]
  \frametitle{树(TREE)}树(Tree)是$n(n \geq 0)$个结点的有限集$T$。 $T$为空时称为空
  树。当$n>0$时,树有且仅有一个特定的称为根(Root)的结点,其余结点可分为$m(m \geq
  0)$个互不相交的子集$T_1, T_2, \cdots, T_m$,其中每个子集又是一棵树,称为子
  树(Subtree)。
  \begin{enumerate}
  \item 各子树是互不相交的集合。
  \item 除根结点,其它结点有唯一前驱。
  \item   一个结点可以有零个或多个后继。
  \end{enumerate}

  \begin{forest}
    [R, for tree={color=white,fill=black}, fill=red!85
    [A [C] [D] [E]]
    [B [F]]
    ]
  \end{forest}
\end{frame}

\begin{frame}[fragile]
  \frametitle{判断哪些是树结构}
  \includegraphics[width=0.3\textwidth]{dot/tree-judge1.pdf} ~~~~~
  \pause
  \includegraphics[width=0.4\textwidth]{dot/tree-judge2.pdf}
\end{frame}

\begin{frame}[fragile]
  \frametitle{判断哪些是树结构}
  \includegraphics[width=0.35\textwidth]{dot/tree-judge3.pdf} ~~~~~
  \pause
  \includegraphics[width=0.4\textwidth]{dot/tree-judge4.pdf}
\end{frame}

\begin{frame}[fragile]
  \frametitle{树的表示形式}
  \includegraphics[width=0.36\textwidth]{dot/tree-represent1.pdf}  \pause
  \scalebox{0.75}{
    \begin{tikzpicture}[b/.style={fill=black!50},n/.style={minimum width=1cm}]
      \draw node[n] (a) {A} node[b, right=0 of a, minimum width=5cm, fill=red!50]{};	

      \draw node[minimum width=0.5cm, below=0.1 of a](bh){} 
      node[n, right=0 of bh] (b) {B} node[b,fill=blue!50, minimum width=4.3cm,right=0 of b]{};	

      \draw node[minimum width=2cm, below=0.2 of bh](dh){} 
      node[n, right=0 of dh] (d) {D} node[b, minimum width=3.5cm,right=0 of d]{};	

      \draw node[minimum width=3.5cm, below=0.2 of dh](ih){} 
      node[n, right=0 of ih] (i) {I} node[b, fill=green!50, minimum width=2.8cm,right=0 of i]{};	

      \draw node[minimum width=3.5cm, below=0.2 of ih](jh){} 
      node[n, right=0 of jh] (j) {J} node[b, fill=green!50, minimum width=2.8cm,right=0 of j]{};	

      \draw node[minimum width=2cm, below=1.2 of dh](eh){} 
      node[n, right=0 of eh] (e) {E} node[b, minimum width=3.5cm,right=0 of e]{};	

      \draw node[minimum width=2cm, below=1.8 of dh](fh){} 
      node[n, right=0 of fh] (f) {F} node[b, minimum width=3.5cm,right=0 of f]{};		


      \draw node[minimum width=0.5cm, below=3.2 of a](ch){} 
      node[n, right=0 of ch] (c) {C} node[b, fill=blue!50, minimum width=4.3cm,right=0 of c]{};	

      \draw node[minimum width=2cm, below=2.8 of dh](gh){} 
      node[n, right=0 of gh] (g) {G} node[b, minimum width=3.5cm,right=0 of g]{};	
      \draw node[minimum width=2cm, below=3.2 of dh](hh){} 
      node[n, right=0 of hh] (h) {H} node[b, minimum width=3.5cm,right=0 of h]{};	

      \draw node[below=0.1 of h] {凹入表表示法};
    \end{tikzpicture}
      } \pause
  \begin{columns}[t]
    \begin{column}{0.4\textwidth}
      \centering
      \vspace{0pt}
      (A(B(D(I,J),E, F),C(G,H)))
      
      广义表表示 \\
      
      \pause
    \end{column}
    \begin{column}{0.5\textwidth}
\scalebox{0.65}{
    \begin{tikzpicture}[n/.style={ellipse,draw}]
      \draw node[n, minimum width=7.8cm, minimum height=3.5cm, fill=red!5]{}
      node[n, minimum width=4.5cm, minimum height=2.5cm, xshift=-1.2cm, fill=blue!5]{}
      node[n, minimum width=2.5cm, minimum height=1.8cm, xshift=2.5cm, fill=blue!5]{}
      node[n, minimum width=2cm, minimum height=1.8cm, xshift=-2cm, fill=yellow!5]{}
      node[n, circle,xshift=-2.5cm, fill=green!5]{I}
      node[n, circle,xshift=-1.7cm, fill=green!5]{J}
      node[n, circle,xshift=-0.5cm, fill=yellow!5]{E}
      node[n, circle,xshift=0.4cm, fill=yellow!5]{F}
      node[n, circle,xshift=2cm, fill=yellow!5]{G}
      node[n, circle,xshift=3cm, fill=yellow!5]{H}
      node[yshift=1.35cm]{A}
      node[xshift=-0.8cm,yshift=0.8cm]{B}
      node[xshift=2.5cm, yshift=0.6cm]{C}
      node[xshift=-2cm,yshift=0.6cm]{D};
      \draw node[yshift=-2.2cm] {嵌套集合表示};
    \end{tikzpicture}
  }
  
    \end{column}
  \end{columns}
\end{frame}

\begin{frame}[fragile]{}
\begin{easylist} \easyitem

\end{easylist}
\end{frame}


