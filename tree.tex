\section{Tree}


\begin{frame}[fragile]{树和二叉树}
  \begin{columns}[T]
    \column{0.5\textwidth}

    树型结构是结点之间有分支,并且具有层次关系的结构,类似于自然界中的树。树有很多
    应用,比如Unix等操作系统中的目录结构。

    \column{0.4\textwidth}
    \includegraphics[width=0.9\textwidth]{imgs/linux-tree.png}
  \end{columns}
\end{frame}

\begin{frame}[fragile]
  \frametitle{例子}
\begin{forest}
 [CEO, for tree={rectangle, minimum width=2cm}, fill=red!10
    [CFO [财务人员] ]
    [CTO [工程师] ]
    [CMO [销售人员] ]
 ]
 \node at (current bounding box.south)
 [below=1ex,draw,cloud,aspect=6,cloud puffs=30]
 {\emph{Simple Company Hierarchy}};
\end{forest}
\end{frame}

\begin{frame}[fragile, plain]
  \scalebox{0.7}{
    \begin{forest}
      [学院, for tree={draw=none, rectangle, minimum width=1cm}, fill=red!10, circle
       [社会学部, grow=west [信息资源管理学院, fill=red!10] [新闻学院] [农业与农村发展学院] [社会与人口学院]
       [公共管理学院] [教育学院]]
       [$\cdots$]
       [ 人文学部,grow=east [哲学院] [文学院] [历史学院] [国学院] [艺术学院] [外国语学院] [清史研究所]]
       ]
       \node at (current bounding box.south)
       [below=1ex,draw,cloud,aspect=6,cloud puffs=30]
       {\emph{人民大学学院设置}};
    \end{forest}
  }
\end{frame}

\begin{frame}[fragile]{内容}
  \begin{easylist} \easyitem
    & 树的基本术语
    & 二叉树
    & 遍历二叉树与线索二叉树
    & 树和森林
    & 哈夫曼树
  \end{easylist}
\end{frame}

\subsection{基本术语}

\begin{frame}[fragile]
  \frametitle{树(TREE)}树(Tree)是$n(n \geq 0)$个结点的有限集$T$。 $T$为空时称为空
  树。当$n>0$时,树有且仅有一个特定的称为根(Root)的结点,其余结点可分为$m(m \geq
  0)$个互不相交的子集$T_1, T_2, \cdots, T_m$,其中每个子集又是一棵树,称为子
  树(Subtree)。
  \begin{columns}[T]
    \column{0.55\textwidth}
    \begin{enumerate}
    \item 各子树是互不相交的集合。
    \item 除根结点,其它结点有唯一前驱。
    \item   一个结点可以有零个或多个后继。
    \end{enumerate}

    \column{0.4\textwidth}
    \begin{forest}
      [R, for tree={color=white,fill=black}, fill=red!85
      [A [C] [D] [E]]
      [B [F]]
      ]
    \end{forest}
  \end{columns}
\end{frame}

\begin{frame}[fragile]
  \frametitle{判断哪些是树结构}
  \includegraphics[width=0.3\textwidth]{dot/tree-judge1.pdf} ~~~~~
  \pause
  \includegraphics[width=0.4\textwidth]{dot/tree-judge2.pdf}
\end{frame}

\begin{frame}[fragile]
  \frametitle{判断哪些是树结构}
  \includegraphics[width=0.35\textwidth]{dot/tree-judge3.pdf} ~~~~~
  \pause
  \includegraphics[width=0.4\textwidth]{dot/tree-judge4.pdf}
\end{frame}

\begin{frame}[fragile]
  \frametitle{树的表示形式}
  \includegraphics[width=0.36\textwidth]{dot/tree-represent1.pdf}  \pause
  \scalebox{0.75}{
    \begin{tikzpicture}[b/.style={fill=black!50},n/.style={minimum width=1cm}]
      \draw node[n] (a) {A} node[b, right=0 of a, minimum width=5cm, fill=red!50]{};	

      \draw node[minimum width=0.5cm, below=0.1 of a](bh){} 
      node[n, right=0 of bh] (b) {B} node[b,fill=blue!50, minimum width=4.3cm,right=0 of b]{};	

      \draw node[minimum width=2cm, below=0.2 of bh](dh){} 
      node[n, right=0 of dh] (d) {D} node[b, minimum width=3.5cm,right=0 of d]{};	

      \draw node[minimum width=3.5cm, below=0.2 of dh](ih){} 
      node[n, right=0 of ih] (i) {I} node[b, fill=green!50, minimum width=2.8cm,right=0 of i]{};	

      \draw node[minimum width=3.5cm, below=0.2 of ih](jh){} 
      node[n, right=0 of jh] (j) {J} node[b, fill=green!50, minimum width=2.8cm,right=0 of j]{};	

      \draw node[minimum width=2cm, below=1.2 of dh](eh){} 
      node[n, right=0 of eh] (e) {E} node[b, minimum width=3.5cm,right=0 of e]{};	

      \draw node[minimum width=2cm, below=1.8 of dh](fh){} 
      node[n, right=0 of fh] (f) {F} node[b, minimum width=3.5cm,right=0 of f]{};		


      \draw node[minimum width=0.5cm, below=3.2 of a](ch){} 
      node[n, right=0 of ch] (c) {C} node[b, fill=blue!50, minimum width=4.3cm,right=0 of c]{};	

      \draw node[minimum width=2cm, below=2.8 of dh](gh){} 
      node[n, right=0 of gh] (g) {G} node[b, minimum width=3.5cm,right=0 of g]{};	
      \draw node[minimum width=2cm, below=3.2 of dh](hh){} 
      node[n, right=0 of hh] (h) {H} node[b, minimum width=3.5cm,right=0 of h]{};	

      \draw node[below=0.1 of h] {凹入表表示法};
    \end{tikzpicture}
  } \pause
  
  \begin{columns}[T]
    \begin{column}{0.4\textwidth}
      \centering
      \vspace{2cm}
      (A(B(D(I,J),E, F),C(G,H)))
      
      广义表表示 \\
      
      \pause
    \end{column}
    \begin{column}{0.5\textwidth}
\scalebox{0.65}{
    \begin{tikzpicture}[n/.style={ellipse,draw}]
      \draw node[n, minimum width=7.8cm, minimum height=3.5cm, fill=red!5]{}
      node[n, minimum width=4.5cm, minimum height=2.5cm, xshift=-1.2cm, fill=blue!5]{}
      node[n, minimum width=2.5cm, minimum height=1.8cm, xshift=2.5cm, fill=blue!5]{}
      node[n, minimum width=2cm, minimum height=1.8cm, xshift=-2cm, fill=yellow!5]{}
      node[n, circle,xshift=-2.5cm, fill=green!5]{I}
      node[n, circle,xshift=-1.7cm, fill=green!5]{J}
      node[n, circle,xshift=-0.5cm, fill=yellow!5]{E}
      node[n, circle,xshift=0.4cm, fill=yellow!5]{F}
      node[n, circle,xshift=2cm, fill=yellow!5]{G}
      node[n, circle,xshift=3cm, fill=yellow!5]{H}
      node[yshift=1.35cm]{A}
      node[xshift=-0.8cm,yshift=0.8cm]{B}
      node[xshift=2.5cm, yshift=0.6cm]{C}
      node[xshift=-2cm,yshift=0.6cm]{D};
      \draw node[yshift=-2.2cm] {嵌套集合表示};
    \end{tikzpicture}
  }
  
    \end{column}
  \end{columns}
\end{frame}

\begin{frame}[fragile]
  \frametitle{基本术语}
  \begin{columns}[t]
    \begin{column}{0.5\textwidth}
      \begin{itemize}
      \item 树(tree)
      \item 子树(sub-tree)
      \item 结点(node)
      \item 结点的度(degree)
      \item 叶子(leaf)
      \item 孩子(child)
      \item 父亲(parents)
      \item 兄弟(sibling)
      \item 祖先
      \item 子孙
      \end{itemize}
    \end{column}
    \begin{column}{0.5\textwidth}
      \begin{itemize}
      \item 树的度(degree)
      \item 结点的层次(level)
      \item 树的深度(depth)
      \item 有序树
      \item 无序树
      \item 森林
      \end{itemize}

      \begin{forest}
        [R
        [A, 
        [C] [D]]
        [B [E]]
        ]
      \end{forest}
    \end{column}
  \end{columns}
\end{frame}

\begin{frame}[fragile, plain]
~
\end{frame}

\subsection{二叉树}
\begin{frame}[fragile]
  \frametitle{二叉树(Binary Tree)}
  \begin{itemize}
  \item 二叉树是一种树型结构,它的每个结点至多只有两个子树,分别称为左子树和右子树。
    二叉树是有序树。
  \item 二叉树是$n(n \geq 0)$个结点构成的有限集合。二叉树或为空,或是由一个根结点及
    两棵互不相交的左右子树组成,并且左右子树都是二叉树。
  \item 在二叉树中要区分左子树和右子树,即使只有一棵子树。这是二叉树与树的最主要的
    差别。
  \end{itemize}

  二叉树的一个重要应用是在查找中的应用。当然,它还 有许多与搜索无关的重要应用,比如
  在编译器的设计领域。
\end{frame}

\begin{frame}[fragile]
  \frametitle{二叉树的五种形态}
  \begin{enumerate}
  \item 空二叉树;
  \item 只有根结点(左右子树都为空);
  \item 只有左子树(右子树为空);
  \item 只有右子树(左子树为空);
  \item 左右子树均不空。
  \end{enumerate}

  \scalebox{0.8}{
    \begin{tikzpicture}[n/.style={draw=black!80, thick,fill=black!80, circle, minimum size=0.6cm},
      t/.style={draw=black!80, ellipse, minimum height=2cm,minimum width=1cm, fill=black!20}]
      \draw node[n,dotted, fill=yellow!1] (t1) at (0,0) {};
      \draw[draw=red] (0.5,0.5) -- (-0.5,-0.5);

      \draw node[n] (t2) at (2,0) {};

      \draw node[n] (t3) at (4.5,0) {} node[t, below left=of t3,xshift=0.8cm, rotate=-30] (t31) {};
      \draw[draw] (t3) -- (t31);

      \draw node[n] (t4) at (6.5,0) {} node[t, below right=of t4,xshift=-0.8cm, rotate=30] (t41) {};
      \draw[draw] (t4) -- (t41);

      \draw node[n] (t5) at (11,0) {} node[t, below left=of t5,xshift=0.8cm, rotate=-30] (t51) {} node[t, below right=of t5,xshift=-0.8cm, rotate=30] (t52) {};
      \draw[draw] (t5) -- (t51);
      \draw[draw] (t5) -- (t52);
    \end{tikzpicture}
  }
\end{frame}

\begin{frame}[fragile]
  \frametitle{请观察二叉树, 并回答下列问题}

  \scalebox{0.7} {
    \begin{forest}
      [A, name=A
      [B, name=B [D [H] [I]] [E [J] [K]]]
      [C, name=C [F [L] [M]] [G [N] [O]]]
      ]
     \draw[draw, dotted, thick] (B.north) ++(-2cm, 0.2cm)-- ++(9cm, 0cm)
     node[above, right, yshift=0.5cm]{1};
     \draw[draw, dotted, thick] (B.south) ++(-2cm, -0.2cm)-- ++(9cm, 0cm)
     node[above, right, yshift=0.5cm]{2};
     \draw[draw, dotted, thick] (B.south) ++(-2cm, -1.6cm)-- ++(9cm, 0cm)
     node[above, right, yshift=0.5cm]{3} node[below, right, yshift=-0.5cm] {4};
    \end{forest}
  }

  \begin{enumerate}
  \item 二叉树的第$i$层最多有多少个结点?
  \item 二叉树深度为k,则它最多有多少个结点?
  \item 二叉树有n个节点,请问它最小深度是几?
  \item 二叉树叶子的数目和度为2的节点的数目是否相等?如果不等,又是什么关系?
  \end{enumerate}
\end{frame}

\begin{frame}[fragile]
  \frametitle{二叉树的性质}
  \begin{itemize}
  \item 性质1: 二叉树的第$i$层至多有$2^{i-1}$个结点。
  \item 性质2: 深度为$k$的二叉树至多有$2^k-1$个结点($k \geq 1$)。
  \item \color{red} 性质3: 二叉树中终端结点数为$n_0$,度为2的结点数为$n_2$,则有$n_0 = n_2 + 1$
    (试证明)
  \end{itemize}
\end{frame}

\begin{frame}[fragile]
  \frametitle{二叉树的性质}
  二叉树中终端结点数为$n_0$,度为2的结点数为$n_2$,则有$n_0 = n_2 + 1$

  \begin{columns}[T,c]
    \begin{column}{0.6\linewidth}
      \begin{itemize}
      \item 设二叉树中度为1的结点数为$n_1$, 二叉树中总结点数为$N$,则有:
        
        $N = n_0 + n_1 + n_2$
      \item 再考虑二叉树中的分支数(每个节点有唯一一个入的分支,根节点除外;再考虑出
        的分支数量), 则有:

        $N - 1 =n_1 + 2 \times n_2$
      \item 整理可得:

        $n_0 = n_2 + 1$
      \end{itemize}
    \end{column}
    \begin{column}{0.35\linewidth}      
      \begin{forest}
        [{}, name=root,for tree={color=white,fill=black}
        [{}, name=n21, grow=-55 [{}, name=n31 [{~}] [{~}]]]
        [{}, name=n22, grow=235 [{}, name=n32,[{~}, draw=none,fill=none, no edge] [{~}]]]
        ]
      \end{forest}
    \end{column}
  \end{columns}
\end{frame}

\begin{frame}[fragile]
  \frametitle{完全二叉树}
  \begin{columns}[T]
    \column{0.55\textwidth}
    \begin{forest}
      [1, for tree={font=\scriptsize}
      [2 [4 [8] [9]] [5 [10] [11]]]
      [3 [6 [12] [{}, no edge,draw=none]] [7]]
      ]\
    \end{forest}
    \pause
    
    \column{0.4\textwidth}
    是否完全二叉树?

    \begin{forest}
      [1
      [2 [4] [5]]
      [3 [{}, no edge, draw=none] [7]]
      ]
    \end{forest}

    \pause
    
    \begin{forest}
      [1
      [2]
      [3 [6] [7]]
      ]
    \end{forest}
  \end{columns}
\end{frame}

\begin{frame}[fragile]
  \frametitle{试找出非完全二叉树}

  \begin{forest}
    [{}, for tree={fill=black}]
  \end{forest} \hspace{3cm}
  \begin{forest}
    [{}, for tree={fill=black} [{}] [{}]]
  \end{forest}
  
  \begin{forest}
    [{}, for tree={fill=black}
    [{} [{}] [{}]]
    [{}]]
  \end{forest}
  \begin{forest}
    [{}, for tree={fill=black}
    [{} [{} [{}] [{}]] [{}]]
    [{}]]
  \end{forest}
  \begin{forest}
    [{}, for tree={fill=black}
    [{} [{}] [{}]]
    [{} [{}] [{}]]]
  \end{forest}
  \begin{forest}
    [{}, for tree={fill=black}
    [{} [{} [{} [{}] [{}]] [{}]] [{}]]
    [{}]]
  \end{forest}
  \begin{forest}
    [{}, for tree={fill=black}
    [{} [{} [{}] [{}, no edge, draw=none, fill=none]] [{}]]
    [{} [{}] [{}]]]
  \end{forest}
\end{frame}

\begin{frame}[fragile]
  \frametitle{二叉树的性质}
  \begin{itemize}
  \item 性质4:具有$n$个结点的完全二叉树的深度为:
    \[ \biggl\lfloor {log_2 n} \biggr\rfloor + 1 \] 
  \end{itemize}
  \small
  对于完全二叉树,设深度为$k$, 由$2^{k-1}-1 < n \leq 2^k - 1$可知,
  $2^{k-1} \leq n <2^k$, 则 $k-1 \leq log_2n < k$
  (参考性质2)
  
  \begin{columns}[T]
    \column{0.45\textwidth}
    \scalebox{0.6}{
      \begin{forest}
        [1, for tree={font=\scriptsize}
        [2 [4 [8] [9]] [5 [10] [11]]]
        [3 [6 [12] [13]] [7 [14] [15]]]
        ]
      \end{forest}
    }
    
    满二叉树
    \column{0.45\textwidth}
    \scalebox{0.6}{
      \begin{forest}
        [1, for tree={font=\scriptsize}
        [2 [4 [8] [9]] [5 [10] [11]]]
        [3 [6 [12] [{}, no edge, draw=none]] [7]]
        ]
      \end{forest}
    }
    
    完全二叉树
  \end{columns}
\end{frame}

\begin{frame}[fragile]
  \frametitle{测试}

\begin{columns}[T]
  \column{0.6\textwidth}
    对于完全二叉树:
  \begin{enumerate}
  \item 若完全二叉树有叶子结点出现在第$k$层,它可能还有(~~~~~~)层的叶子结点;
  \item 若某结点的右子树的最大层次为$L$,则其左子树的最大层次为(~~~~~~~~);
  \item 若按如图所示的编号方式,试求出编号为$i$的节点的父节点和子节点的编号.
  \end{enumerate}

  \column{0.36\textwidth}
  \scalebox{0.6}{
    \begin{forest}
      [1
      [2 [4 [8] [9]] [5 [10] [11]]]
      [3 [6 [12] [{}, no edge,draw=none]] [7]]
      ]\
    \end{forest}
  }
\end{columns}

\pause
\begin{enumerate}
\item $k-1$, $k+1$
\item $L$ or $L+1$
\end{enumerate}
\end{frame}


\begin{frame}[fragile]
  \frametitle{二叉树的性质}
  \begin{center}
    \scalebox{0.6} {
      \begin{forest}
        [1, name=A,for tree={}
        [2, name=B [4 [8] [9]] [5 [10] [11]]]
        [3, name=C [6 [12] [{}, no edge,draw=none] ] [7] ]
        ]
        \draw[draw, dotted, thick] (B.north) ++(-2cm, 0.2cm)-- ++(9cm, 0cm)
        node[above, right, yshift=0.5cm]{1};
        \draw[draw, dotted, thick] (B.south) ++(-2cm, -0.2cm)-- ++(9cm, 0cm)
        node[above, right, yshift=0.5cm]{2};
        \draw[draw, dotted, thick] (B.south) ++(-2cm, -1.6cm)-- ++(9cm, 0cm)
        node[above, right, yshift=0.5cm]{3} node[below, right, yshift=-0.5cm] {4};
      \end{forest}
    }
  \end{center}

  \begin{itemize}
  \item 性质5: 对具有$n$个结点的完全二叉树的结点按层次顺序编号,对任意结点$i$有:
    \begin{itemize}
    \item (有关结点$i$的双亲)若$i=1$,则为二叉树的根结点, 没有双亲;否则双亲结点的
      编号为: $\biggl\lfloor{\dfrac{i}{2}}\biggr\rfloor$
    \item (有关结点$i$的孩子) 若$n<2 \cdot i$, 则结点$i$无左孩子;否则左孩子编号
      是$2 \cdot i$。
    \item 若$n<2 \cdot i+1$,则结点i无右孩子;否则右孩子编号为$2 \cdot i+1$
    \end{itemize}
  \end{itemize}
\end{frame}

\begin{frame}[fragile]
  \frametitle{二叉树的存储结构:顺序存储}
  \begin{columns}
    \column{0.45\textwidth}

    \scalebox{0.6} {
      \begin{forest}
        [1, name=A,for tree={color=white,fill=black}
        [2, for tree={dotted,color=black,fill=white}, [4 [8] [9]] [5 [10] [11]]]
        [3, name=C [6 [12] [{}, no edge,draw=none, fill=none] ] [7] ]
        ]
      \end{forest}
    }

    \vspace{2cm}
    \scalebox{0.6}{
      \begin{tikzpicture}[b/.style={draw, fill=black, white, minimum size=0.8cm}, n/.style={draw, minimum size=0.8cm}]
        \draw node[b] (n1) {1} node[n, right=0 of n1](n2) {2}  node[b, right=0 of n2](n3) {3}
        node[n, right=0 of n3] (n4) {4} node[n, right=0 of n4](n5) {5}  node[b, right=0 of n5](n6) {6}
        node[b, right=0 of n6] (n7) {7} node[n, right=0 of n7](n8) {8}  node[n, right=0 of n8](n9) {9}
        node[n, right=0 of n9] (n10) {10} node[n, right=0 of n10](n11) {11}  node[b, right=0 of n11](n12) {12} ;
      \end{tikzpicture}
    }

    \column{0.55\textwidth}
    \begin{minted}[bgcolor=yellow!20, fontsize=\scriptsize]{c}
      #define MAX_SIZE 100
      typedef int SqBiTree[MAX_SIZE];
      SqBiTree bt;
    \end{minted}

    \begin{minted}[bgcolor=red!5, fontsize=\scriptsize]{java}
      class SqBiTree {
        //static int MAX_SIZE = 100;
        //int[] data = new int[MAX_SIZE];
        List<Integer> data = new ArrayList<Integer>();
      }
    \end{minted}
  \end{columns}
  
  \begin{itemize}
  \item 把结点安排成一个恰当的序列(编号),存储在数组中
  \item 便于“随机存取”
  \end{itemize}
\end{frame}

\begin{frame}[fragile]
  \frametitle{二叉树的存储结构:顺序存储}
  \begin{itemize}
  \item 适用于完全二叉树
  \end{itemize}

  \begin{columns}
    \column{0.6\textwidth}
    \scalebox{0.8} {
      \begin{forest}
        [$a$, for tree={color=white,fill=black}
        [{}, draw=none, no edge, fill=none]
        [$b$
        [{}, draw=none, no edge, fill=none]
        [$c$ [{}, draw=none, no edge, fill=none] [$d$] ]
        ]
        ]
      \end{forest}
    }

    \column{0.4\textwidth}
    \scalebox{0.8} {
      \begin{forest}
        [1, for tree={color=white,fill=black}
        [2 [4] [5 [6] [7]]]
        [3]
        ]
      \end{forest}
    }
  \end{columns}

  \pause
  
  \hspace{-1cm}
  \scalebox{0.7}{
    \begin{tikzpicture}[ n/.style={minimum size=0.8cm}]
      \draw node[n] (n0) {0};
      \foreach \x  [evaluate = \x as \xp using int(\x-1)] in {1, ..., 14}
      \draw node[n, right=0 of n\xp](n\x) {$\x$};

      \foreach \x in {1,3,4,5,7,8, ..., 13}
      \draw node[n, above=0 of n\x, draw](d\x) {$0$};

      \foreach \x/\y in {0/a,2/b,6/c,14/d}
      \draw node[n, above=0 of n\x, draw, fill=red!10](d\x) {$\y$};
    \end{tikzpicture}
  }

  \pause
  \hspace{4cm}
  \scalebox{0.7}{
    \begin{tikzpicture}[ n/.style={minimum size=0.8cm}]
      \draw node[n] (n0) {0};
      \foreach \x  [evaluate = \x as \xp using int(\x-1)] in {1, ..., 10}
      \draw node[n, right=0 of n\xp](n\x) {$\x$};

      \foreach \x in {5, ..., 8}
      \draw node[n, above=0 of n\x, draw](d\x) {$0$};

      \foreach \x/\y in {0/1,1/2,2/3,3/4,4/5,9/6,10/7}
      \draw node[n, above=0 of n\x, draw, fill=red!10](d\x) {$\y$};
    \end{tikzpicture}
  }
\end{frame}

\begin{frame}[fragile]
  \frametitle{二叉树的链式存储:二叉链表}
  \begin{columns}[T]
    \column{0.4\textwidth}
    \scalebox{0.7}{
      \begin{forest}
        [1
        [2 [{}, no edge, draw=none] [4 [5] [6]]]
        [3]]
      \end{forest}
    }
    \column{0.4\textwidth}
    \includegraphics[width=0.6\textwidth]{dot/tree-store-bilink.pdf}
  \end{columns}

  \small
  \begin{columns}
    \column{0.48\textwidth}
    \begin{minted}[fontsize=\scriptsize,bgcolor=red!5]{c}
      typedef struct BiTNode  { // C Code
        ElemType data;
        struct BiTNode *lchild, *rchild;
      } BiTNode, *BiTree;
    \end{minted}

    \column{0.48\textwidth}
    \begin{minted}[fontsize=\scriptsize,bgcolor=yellow!10]{c}
      class BiTNode<T> { //Java Code
        T data;
        BiTNode lchild, rchild;
      }
    \end{minted}
  \end{columns}

  思考:含$n$个结点的二叉链表中有多少个空指针?

  \pause
  指针域一共有$2*n$个,分支共有$N-1$,  每个分支占用一个指针域,所以空指针数量为:  $2*n - (n-1) = n+1$
\end{frame}

\begin{frame}[fragile]
  \frametitle{二叉树的链式存储:三叉链表}
  \begin{itemize}
  \item 二叉链表不便查找父节点, 可加一个指向双亲的指针
  \end{itemize}

  \begin{columns}[T]
    \column{0.5\textwidth}
    \scalebox{0.5}{
      \begin{forest}
        [1
        [2 [{}, no edge, draw=none] [4 [5] [6]]]
        [3]]
      \end{forest}
    }
    \includegraphics[width=0.8\textwidth]{dot/tree-store-trilink.pdf}
    
    \column{0.5\textwidth}
    \begin{minted}[fontsize=\scriptsize,bgcolor=red!5]{c}
      typedef struct BiTNode  { // C Code
        ElemType data;
        struct BiTNode *lchild, *rchild, *parent;
      } BiTNode, *BiTree;
    \end{minted}

    \begin{minted}[fontsize=\scriptsize,bgcolor=yellow!10]{c}
      class BiTNode<T> { //Java Code
        T data;
        BiTNode lchild, rchild, parent;
      }
    \end{minted}
  \end{columns}

  思考: 有n个结点的三叉链表有多少个空指针?

  \pause
  \scriptsize
  对于二叉链表,指针域一共有$2*n$个,分支共有$N-1$,  每个分支占用一个指针域,所以空指针数量为$2*n - (n - 1) = n+1 $;  对于parent,根节点无父节点,所以共有$n+1+1=n+2$个空指针
\end{frame}

\begin{frame}[fragile, plain]
  ~  
\end{frame}


\subsection{遍历二叉树和线索二叉树}
\begin{frame}[fragile]
  \frametitle{遍历二叉树和线索二叉树}

  在二叉树的一些应用中,常常要求在树中查找具有某种特征的结点,或者对树中全部结点逐
  一进行某种处理。这就引入了遍历二叉树的问题,即如何按某条搜索路径巡访树中的每一个
  结点,使得每一个结点均被访问一次且仅访问一次。
\end{frame}

\begin{frame}[fragile]
  \frametitle{二叉树的遍历}
 \begin{itemize}
 \item 遍历是指按某种方式访问所有结点,使每个结点被访问一次且只被访问一次。
 \item 二叉树的遍历是按一定规则将二叉树的结点排成一个线性序列,即非线性序列线性
   化。
 \item 遍历的方式: 深度优先和广度优先,深度优先又分为三种:
   \begin{itemize}
   \item 先序次序
   \item 中序次序(对称序次序)
   \item 后序次序 
   \end{itemize}
 \end{itemize} 
\end{frame}

\begin{frame}[fragile]
  \frametitle{二叉树的遍历}
  \begin{easylist}
    & 1、先序遍历二叉树的操作定义为: 若二叉树为空,则空操作;否则
    && (1)访问根结点;
    && (2)先序遍历左子树;
    && (3)先序遍历右子树。
    & 2、中序遍历二叉树的操作定义为:若二叉树为空,则空操作;否则
    && (1)中序遍历左子树;
    && (2)访问根结点;
    && (3)中序遍历右子树。
    & 3、后序遍历二叉树的操作定义为:若二叉树为空,则空操作;否则
    && (1)后序遍历左子树;
    && (2)后序遍历右子树;
    && (3)访问根结点。
  \end{easylist}
\end{frame}

\begin{frame}[fragile]
  \frametitle{示例}
  \scalebox{0.8}{
    \begin{forest}
      [A
      [B [D] [{}, no edge, draw=none]]
      [C [E [{}, no edge, draw=none] [G]] [F [H] [I]]]
      ]
    \end{forest}
  }

  \pause
  \begin{itemize}
  \item 先序:ABDCEGFHI
  \item 中序:DBAEGCHFI
  \item 后序:DBGEHIFCA
  \item 广度优先:ABCDEFGHI
  \end{itemize}
\end{frame}

\begin{frame}[fragile]
  \frametitle{二叉树的遍历}
  \begin{columns}[T]
    \column{0.6\textwidth}
    //先序遍历二叉树递归算法C伪代码
    \begin{minted}[bgcolor=red!5, fontsize=\small]{c}
      status preOrderTraverse(BiTree T){
        if(T){
          printf(T->data);
          preOrderTraverse(T->lchild);
          preOrderTraverse(T->rchild);
        }
      }
    \end{minted}
    \begin{itemize}
    \item status代表什么?
    \item printf代表什么?
    \end{itemize}

    \column{0.4\textwidth}
    \scalebox{0.8}{
      \begin{forest}
        [A
        [B [D] [{}, no edge, draw=none]]
        [C [E [{}, no edge, draw=none] [G]] [F [H] [I]]]
        ]
      \end{forest}
    }    
  \end{columns}
\end{frame}


\begin{frame}[fragile]
  \frametitle{二叉树的遍历}
  \begin{columns}[T]
    \column{0.6\textwidth}
    先序遍历递归算法的Java实现
  
    \begin{minted}[bgcolor=yellow!10,fontsize=\small]{java}
      class Node {
        String data;
        Node left, right;
      }
      class Tree {
        void preOrderTraverse(Node node) {
          if(node != null) {
            System.out.println(node.data);
            preOrderTraverse(node.left);
            preOrderTraverse(node.right);
          }
        }
      }
    \end{minted}

    \column{0.4\textwidth}
    \scalebox{0.8}{
      \begin{forest}
        [A
        [B [D] [{}, no edge, draw=none]]
        [C [E [{}, no edge, draw=none] [G]] [F [H] [I]]]
        ]
      \end{forest}
    }
  \end{columns}
\end{frame}

\begin{frame}[fragile]
  \frametitle{如下将得到树的什么序列?}
  \begin{columns}[T]
    \column{0.6\textwidth}
    \begin{minted}[bgcolor=yellow!10,fontsize=\small]{c}
      status traverse(BiTree T) {
        InitStack(S);
        p = T;
        while(p || !StackEmpty(S)) {
          if(p) {
            push(S, p); p = p->lchild;
          } else {
            pop(S, p);
            printf(p->data);
            p = p->rchild;
          }
        }
        return OK;
      }
    \end{minted}

    \column{0.4\textwidth}
    \scalebox{0.8}{
      \begin{forest}
        [A
        [B [D] [{}, no edge, draw=none]]
        [C [E [{}, no edge, draw=none] [G]] [F [H] [I]]]
        ]
      \end{forest}
    }
  \end{columns}  
\end{frame}

\begin{frame}[fragile]
  \frametitle{二叉树的遍历:广度优先}
  \begin{columns}[T]
    \column{0.6\textwidth}
    算法步骤:
    \begin{itemize}
    \item 访问节点
    \item 从左到右依次访问儿子节点
    \item 重复前一步骤
    \end{itemize}
    
    \begin{tikzpicture}[ n/.style={minimum width=1cm,minimum height=0.6cm, draw}]
      \draw node[] (n0) {};
      \foreach \x  [evaluate = \x as \xp using int(\x-1)] in {1, ..., 7}
      \draw node[n, right=0 of n\xp](n\x) {};
      
      \draw node[n, right=0 of n0] {A};
    \end{tikzpicture}

    \begin{tikzpicture}[ n/.style={minimum width=1cm,minimum height=0.6cm, draw}]
      \draw node[] (n0) {};
      \foreach \x  [evaluate = \x as \xp using int(\x-1)] in {1, ..., 7}
      \draw node[n, right=0 of n\xp](n\x) {};
      
      \draw node[n, right=0 of n1] {B} node[n, right=0 of n2]{C};
    \end{tikzpicture}

    \begin{tikzpicture}[ n/.style={minimum width=1cm,minimum height=0.6cm, draw}]
      \draw node[] (n0) {};
      \foreach \x  [evaluate = \x as \xp using int(\x-1)] in {1, ..., 7}
      \draw node[n, right=0 of n\xp](n\x) {};
      
      \draw node[n, right=0 of n2] {C} node[n, right=0 of n3]{D};
    \end{tikzpicture}

    .......

    课堂练习:编程实现广度优先遍历。
    
    \column{0.4\textwidth}
    \scalebox{0.8}{
      \begin{forest}
        [A
        [B [D] [{}, no edge, draw=none]]
        [C [E [{}, no edge, draw=none] [G]] [F [H] [I]]]
        ]
      \end{forest}
    }
  \end{columns}
  
\end{frame}

\begin{frame}[fragile]
  \frametitle{补充}
  \begin{itemize}
  \item 由一棵给定的二叉树可以获得三种遍历序列,同样,也可以由这些遍历序列来重新构
    造二叉树。
  \item 举例
    
    先序:ABCDEFGHIJ
    
    中序:CBDEAFHIGJ
  \item 由于不能从遍历序列中区分二叉树的左、右子树,因此单用一个遍历序列是无法构造
    二叉树的。利用中序遍历序列,并结合先序遍历序列或后序遍历序列就能重新构造二叉
    树。
  \end{itemize}
\end{frame}
